\huge
\scshape\color{AzulFI_dark}
\Bold{Resumen}\\

\normalfont\normalsize\normalcolor
Este trabajo consiste del diseño y fabricación de una placa de circuito impreso o PCB (del inglés \textit{Printed Circuit Board}) para el convertidor de potencia aislado de una plataforma experimental de evaluación de sistemas híbridos. Se debe implementar un circuito de alta complejidad, de más de 150 componentes discretos, en una PCB de doble capa (doble faz) de dimensiones reducidas. Para estas tareas, se utilizó la herramienta de automatización de diseño electrónico (EDA, del inglés \textit{Electronic Design Automation}) KiCad, un software libre y gratuito de nivel profesional.\\

%\vspace{1cm}
%\huge
%\scshape\color{AzulFI_dark}
%\Bold{Abstract}\\

%\normalsize\normalfont\normalcolor
%This work entails the study, design, implementation and validation of an experimental platform for the evaluation of hybrid energy generation systems based on Proton Exchange Membrane Fuel Cells (PEMFC). This platform incorporates a full-bridge isolated switched-mode DC-DC electronic converter, monitored via the measurement of its state variables, and controlled by a pulse-width modulated (PWM) signal, generated using a high-performance Digital Signal Controller (DSC). This converter provides the adaptation from the variable output voltage of the PEMFC to the fixed voltage of the common DC bus at the system output.\\

%This report details the process through which the goals were achieved: study and understanding of the different DC-DC converter topologies, simulation of the selected converter topology using circuit simulation tools, design process of auxiliary circuits, including driver, sensing and protection circuits,  implementation of the system PCB (printed circuit board) through the use of electronic design automation (EDA) software, programming of system control algorithms, and experimental validation of the working platform.\\ 