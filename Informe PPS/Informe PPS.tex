\documentclass[10pt,twoside]{article}                           % Twoside para poder poner números en lados alternados

\usepackage[spanish,es-tabla]{babel}                                     % Configuración de lenguaje
\usepackage[a4paper,top=2cm,bottom=2cm,left=3cm,right=3cm,marginparwidth=1.75cm]{geometry}          % Configurar tamaño de pagina y márgenes
%\usepackage[a4paper, top=2cm, bottom=2cm, outer=0cm, inner=0cm]{geometry}
\usepackage{fancyhdr}
\usepackage[autostyle=true]{csquotes}
\usepackage[backend=biber,sorting=none]{biblatex}               % Paquete para bibliografía 
\addbibresource{Bibliografia.bib}                               % Llamo al archivo que contiene las referencias
\usepackage{amsmath}
\usepackage{amsfonts}
\usepackage{makecell}                                           % Para poder poner line breaks en tablas
\renewcommand\cellgape{\Gape[4pt]}
\usepackage{caption}                                            % Para que "Figura x.y" este en bold
\usepackage{eso-pic}                                            % Para poder poner imágenes de fondo
\usepackage{multicol}                                           % Para multiples columnas
\usepackage{array, multirow}
\usepackage{moresize}                                           % Para tener texto de tamaño \HUGE
\usepackage{graphicx}
\usepackage{lipsum}
\usepackage[hidelinks]{hyperref}                                % Para links entre partes del documento (hidelinks no le cambia el color a los links)
\usepackage{float}                                              % Permite forzar una posición para una figura
\usepackage[justification=centering,labelsep=space]{caption}    % Para poder centrar captions de figuras
\usepackage{gensymb}                                            % Símbolo de grados (\degree)
\usepackage{subfigure}                                          % Para poder usar subfiguras (logos de UNLP y FI juntos)
\usepackage{listings}                                           % Para poder incluir código
\usepackage{xcolor}                                             % Para poder definir colores  
\usepackage{enumitem}                                           % Para poder poner letras como items de listas
\usepackage{afterpage}                                          % Para poder insertar pagina en blanco
\usepackage{titlesec}                                           % Para cambiar tamaño de headers de secciones, subsecciones, etc.
\usepackage{titletoc}                                           % Para modificar el aspecto del índice (Table of Contents - ToC)
\usepackage[pagecolor=none]{pagecolor}                          % Para permitir usar colores de fondo en páginas
\usepackage{siunitx}                                            % Para notación científica (y unidades SI)
\usepackage{mathspec}                                           % Para tipografías custom, incluyendo ecuaciones (Incluye fontspec)

%-------------------------------------------------------------------------------------------------------%

% Definir todas las tipografías de mathspec

\setmainfont{Montserrat}                                        % Letra principal
\newfontfamily\Thin{Montserrat Thin}                            % Defino todos los pesos, también se pueden definir tipografías adicionales de la misma manera
\newfontfamily\ExtraLight{Montserrat ExtraLight}
\newfontfamily\Light{Montserrat Light}
\newfontfamily\Medium{Montserrat Medium}
\newfontfamily\SemiBold{Montserrat SemiBold}
\newfontfamily\Bold{Montserrat Bold}
\setmonofont[Scale=1.1]{Fraunces 72pt Soft SemiBold}
\setmathrm[BoldFont = {Montserrat SemiBold Italic}]{Montserrat} 
\setmathfont(Digits,Latin){Montserrat}                          % Tipografía de matemática

% Comando para linea horizontal configurable que llene todo el espacio disponible 

\newcommand{\xfill}[2][1ex]{{%                                  % Si no hay texto antes del \xfill, hay que agregarle algun caracter como "\tiny\ " para que compile
  \dimen0=#2\advance\dimen0 by #1
  \leaders\hrule height \dimen0 depth -#1\hfill%
}}

% Comando para pagina en blanco sin numero de hoja

\newcommand{\ncblankpage}
{                          
    \null
    \thispagestyle{empty}
    \addtocounter{page}{-1}                         % No aumenta el contador de páginas
    \newpage
}



\newcommand{\blankpage}                             % Aumenta el contador de páginas
{                          
    \null
    \thispagestyle{empty}
    \newpage
}

% Definición de colores

\definecolor{AzulFI}{rgb}{0, 0.394, 0.645}                  % 0064A5
\definecolor{AzulFI_dark}{rgb}{0, 0.2353, 0.3882}           % 003C63
\definecolor{AzulFI_darker}{rgb}{0, 0.196, 0.3216}          % 003252
\definecolor{AzulFI_light}{rgb}{0.898, 0.9373, 0.9647}      % E5EFF6
%\definecolor{codegreen}{rgb}{0,0.6,0}
%\definecolor{codegray}{rgb}{0.5,0.5,0.5}
%\definecolor{codepurple}{rgb}{0.58,0,0.82}
%\definecolor{backcolour}{rgb}{0.95,0.95,0.92}

% Estilo del header y footer

\pagestyle{fancy}

\renewcommand{\sectionmark}[1]{\markright{#1}}                          % Borra el numero de sección de \rightmark            
\renewcommand{\subsectionmark}[1]{}                                     % No marca las subsecciones

% Estilo nuevo
\fancyfoot[CO]{\SemiBold%
    \rule[-0.2em]{1pt}{1.1em}\hspace{0.5em}%
    \color{AzulFI_dark}Tomás Tavella\hspace{0.5em}\normalcolor\rule[-0.2em]{1pt}{1.1em}%
    \hspace{-0.05em}\xfill[0.6ex]{1pt}\hspace{-0.05em}%
    \rule[-0.2em]{1pt}{1.1em}\color{AzulFI_dark}\hspace{0.5em}\thepage\normalcolor\hspace{0.5em}\rule[-0.2em]{1pt}{1.1em}%
}

\fancyfoot[CE]{\SemiBold%
    \rule[-0.2em]{1pt}{1.1em}\hspace{0.5em}\color{AzulFI_dark}\thepage\hspace{0.5em}\normalcolor\rule[-0.2em]{1pt}{1.1em}%
    \hspace{-0.05em}\xfill[0.6ex]{1pt}\hspace{-0.05em}%
    \rule[-0.2em]{1pt}{1.1em}\color{AzulFI_dark}\hspace{0.5em}Tomás Tavella\normalcolor\hspace{0.5em}\rule[-0.2em]{1pt}{1.1em}%
}

\fancyhead[C]{\SemiBold%
    \rule[-0.2em]{1pt}{1.1em}\hspace{0.5em}\scshape\color{AzulFI_dark}Capítulo \thesection\normalcolor\hspace{0.5em}\rule[-0.2em]{1pt}{1.1em}%
    \hspace{-0.05em}\xfill[0.6ex]{1pt}\hspace{-0.05em}%
    \rule[-0.2em]{1pt}{1.1em}\color{AzulFI_dark}\hspace{0.5em}\rightmark\normalcolor\hspace{0.5em}\rule[-0.2em]{1pt}{1.1em}%
    }

\fancyhead[R,L]{}

\renewcommand{\headrulewidth}{0pt}
\setlength{\headheight}{12.60013pt}

\fancypagestyle{plain}{
    \fancyhead[L,C,R]{}
    %\fancyfoot[C]{}                                                    % Sólo va con el estilo viejo
    \renewcommand{\headrulewidth}{0pt}                                  % Para que no haya linea horizontal de encabezado en las páginas de índice y título de sección
}

% Parametros para setear como se muestra el codigo

\lstdefinestyle{mystyle}{                       
    backgroundcolor=\color{backcolour},
    commentstyle=\color{codegreen},
    keywordstyle=\color{magenta},
    numberstyle=\tiny\color{codegray},
    stringstyle=\color{codepurple},
    basicstyle=\ttfamily\scriptsize,
    breakatwhitespace=false,         
    breaklines=true,                 
    captionpos=b,                    
    keepspaces=true,                 
    numbers=left,                    
    numbersep=5pt,                  
    showspaces=false,                
    showstringspaces=false,
    showtabs=false,                  
    tabsize=4
}

\lstset{style=mystyle}

% Macro para divisores horizontales

\newcommand{\divider}                           
{
\vspace{0.4cm}
\begin{center}
    \tiny\ \xfill[8pt]{1pt}\ \ \linebreak
\end{center}
%\vspace{0.25cm}
\normalsize
}

% Macro para usar colorbox en ecuaciones

\newcommand{\highlight}[1]{\colorbox{AzulFI_light}{$\displaystyle #1$}}

% Macro para comillas

\newcommand{\quotes}[1]{``#1''}                 % Comando para double quotes

% Números de figura y ecuación por seccion (1.1 , 2.1, etc.)

\numberwithin{figure}{section}                  % Numeros de figura por seccion
\numberwithin{equation}{section}                % Numeros de ecuacion por seccion
\numberwithin{table}{section}

% Definir formato de titulos de secciones y subsecciones

\titleformat{\section}[display]                                     % display permite el numero en linea separada del nombre
{\raggedright\color{AzulFI_dark}\scshape\Bold\Huge\centering}       % Formato del título de seccion
{\normalcolor\tiny\ \Huge\xfill[8pt]{1pt}\rule[-0.2em]{1pt}{1.1em}\hspace{0.5em}\color{white}\thesection\normalcolor\hspace{0.5em}\rule[-0.2em]{1pt}{1.1em}\xfill[8pt]{1pt}\tiny\ }{0.5em}{}       % Formato del número de seccion

\titleformat{\subsection}[hang]{\raggedright\color{AzulFI_dark}\scshape\huge\Bold}{\thesubsection\quad}{0.5em}{}                      % Se agrega raggedright a todos para que no separe palabras en dos cuando tiene que hacer un line break

\titleformat{\subsubsection}[hang]{\raggedright\color{AzulFI_dark}\scshape\Large\Bold}{\thesubsubsection\quad}{0.5em}{}
\titleformat{\paragraph}[hang]{\raggedright\color{AzulFI_dark}\large\SemiBold}{}{0.5em}{}
\titleformat*{\subparagraph}{\raggedright\SemiBold}

% Formato de captions de figuras

\DeclareCaptionFormat{custom}
{%
    \SemiBold\scshape\color{AzulFI_dark} #1#2 \normalfont\normalcolor \textit{#3}                     % #1 es el número, #2 el separador y #3 el texto
}
\captionsetup{format=custom}

% Formato de Table of Contents (ToC)

\contentsmargin{1em}
\dottedcontents{section}[1.5em]{\large\vspace{0.4cm}\Bold\scshape\color{AzulFI_dark}}{1.5em}{0pc}      % Secciones en Bold en TableofContents
\dottedcontents{subsection}[4em]{\vspace{0.1cm}\Medium}{2.5em}{0.6pc}
\dottedcontents{subsubsection}[7.5em]{}{3.5em}{0.6pc}

%---------------------------------------------------------------------------------------------------------%
%------------------------------------------ Inicio de Documento ------------------------------------------%
%---------------------------------------------------------------------------------------------------------%

\begin{document}

    \nocite{*}                                                      % Se usa para que aparezcan todas la referencias del .bib sin tener que citarlas en el texto

    \newgeometry{top=1.5cm,bottom=2cm,left=1cm,right=1cm,marginparwidth=1.75cm}
    %\newpagecolor{AzulFI}\afterpage{\restorepagecolor\blankpage}    % Página en blanco entre portada y agradecimientos
    

        \begin{titlepage}
        \begin{center}
            \vspace*{0.5cm}
            \Huge
            \textbf{Diseño y desarrollo de una plataforma experimental de evaluación de sistemas híbridos basados en pilas de combustible}    % Titulo
            \\
            \vspace{0.5cm}
            \huge
            Proyecto Final                                       % Subtitulo
            \\
            \vspace{2cm}
            \Large
            \textbf{Autor:}
            \\
            \large
            \vspace{0.2cm}
            Tomás Tavella - 68371/4
            \\
            \vspace{1cm}
            \Large
            \textbf{Director:}
            \\
            \vspace{0.2cm}
            \large
            Ing. Jorge Anderson Azzano
            \\
            \vspace{0.3cm}
            \Large
            \textbf{Co-director:}
            \\
            \vspace{0.2cm}
            \large
            Dr. Ing. Paul F. Puleston
            \\
            \vfill
            \begin{figure}[H]
                \centering
                \begin{subfigure}
                    \centering
                    \includegraphics[width=0.25\textwidth]{Imagenes/UNLP.pdf}
                \end{subfigure}
                \begin{subfigure}
                    \centering
                    \includegraphics[width=0.32\textwidth]{Imagenes/FI.jpg}
                \end{subfigure}
            \end{figure}
            \vspace{1cm}
            \textit{
            Facultad de Ingeniería
            \\
            Universidad Nacional de La Plata}
            \vspace{1cm}
        \end{center}
    \end{titlepage}
    \restoregeometry\blankpage
    
    \addtocounter{page}{-1}
    \newpage 
    \thispagestyle{empty}                                           % Para que no se muestre el número de página al final (igual contribuye a la cuenta total)
    \afterpage{\ncblankpage}

    \huge
\scshape
\textbf{Resumen}\\

\normalfont\normalsize
Este trabajo consiste del estudio, diseño, implementación y validación de una plataforma experimental para la evaluación de sistemas híbridos de generación energía (SHGE) a partir de pilas o celdas de combustible de tipo PEMFC (\textit{Proton Exchange Membrane Fuel Cell}). Esta plataforma consiste en un sistema de conversión electrónico de tipo CC-CC conmutado y aislado, de topología puente completo; monitoreado mediante la medición de sus estados, y controlado por una excitación de tipo PWM (\textit{Pulse-Width Modulation}) provista por un DSC (\textit{Digital Signal Controller}) de alta performance. Este conversor es requerido para poder adaptar la tensión variable que entrega una celda de combustible a una tensión de salida fija para conectar a un bus común de corriente continua.\\

En el desarrollo de este informe se detallan las tareas realizadas para cumplir este objetivo: el estudio y comprensión de las topologías de conversión CC-CC; la simulación de la topología elegida mediante herramientas de simulación circuitales; el diseño de circuitos auxiliares de excitación, sensado y protección; la implementación del sistema en una placa de circuito impreso mediante software EDA (\textit{Electronic Design Automation}); la programación de los algoritmos de control del sistema; y, finalmente la validación experimental de la plataforma.\\

\vspace{1cm}
\huge
\scshape
\textbf{Abstract}\\

\normalsize\normalfont
This work entails the study, design, implementation and validation of an experimental platform for the evaluation of hybrid energy generation systems based on Proton Exchange Membrane Fuel Cells (PEMFC). This platform incorporates a full-bridge isolated switched-mode DC-DC electronic converter, monitored via the measurement of its state variables, and controlled by a pulse-width modulated (PWM) signal, generated using a high-performance Digital Signal Controller (DSC). This converter provides the adaptation from the variable output voltage of the PEMFC to the fixed voltage of the common DC bus at the system output.\\

This report details the process through which the goals were achieved: study and understanding of the different DC-DC converter topologies, simulation of the selected converter topology using circuit simulation tools, design process of auxiliary circuits, including driver, sensing and protection circuits,  implementation of the system PCB (printed circuit board) through the use of electronic design automation (EDA) software, programming of system control algorithms, and experimental validation of the working platform.\\ 

    \newpage
    \afterpage{\blankpage} 
    \thispagestyle{plain}
    \tableofcontents
    \newpage

    \section{Introducción}

\vspace{0.5cm}

\Large\scshape
Sistema completo en el que se engloba la plataforma de evaluación en estudio. Contexto global como justificación
\normalfont

\divider

Lorem ipsum dolor sit amet, consectetur adipiscing elit, sed do eiusmod tempor incididunt ut labore et dolore magna aliqua. Aliquet enim tortor at auctor urna. Ac orci phasellus egestas tellus rutrum tellus pellentesque eu. Aliquam eleifend mi in nulla. Sit amet cursus sit amet dictum sit amet justo. Tellus orci ac auctor augue mauris augue neque gravida in. Tincidunt dui ut ornare lectus sit amet est. Nulla facilisi morbi tempus iaculis urna id. Quis vel eros donec ac odio tempor orci dapibus. Sed cras ornare arcu dui vivamus. Augue neque gravida in fermentum et. At urna condimentum mattis pellentesque id nibh tortor id. Malesuada fames ac turpis egestas integer eget. Nec feugiat in fermentum posuere urna nec. Pellentesque pulvinar pellentesque habitant morbi. Nunc sed id semper risus in hendrerit gravida.\\

\subsection{Subsección 1}

Parturient montes nascetur ridiculus mus. Pulvinar etiam non quam lacus suspendisse faucibus. Fusce id velit ut tortor pretium viverra suspendisse potenti nullam. Porta non pulvinar neque laoreet suspendisse. Pellentesque id nibh tortor id aliquet lectus. Semper viverra nam libero justo. Vitae tortor condimentum lacinia quis vel eros donec. Ullamcorper velit sed ullamcorper morbi tincidunt. Pellentesque habitant morbi tristique senectus et netus. Non curabitur gravida arcu ac tortor dignissim convallis aenean. Fringilla urna porttitor rhoncus dolor purus non enim praesent. Eget aliquet nibh praesent tristique magna sit amet purus gravida. Orci porta non pulvinar neque. Id porta nibh venenatis cras sed felis. Id neque aliquam vestibulum morbi blandit cursus risus at.\\

\subsection{Subsección 2}

In iaculis nunc sed augue lacus. Odio ut enim blandit volutpat maecenas volutpat. Cras sed felis eget velit aliquet. Risus in hendrerit gravida rutrum quisque non. Risus in hendrerit gravida rutrum quisque non tellus orci. Nec ullamcorper sit amet risus nullam eget felis. Gravida arcu ac tortor dignissim convallis aenean et tortor at. Vehicula ipsum a arcu cursus vitae congue mauris rhoncus. Montes nascetur ridiculus mus mauris vitae ultricies leo integer malesuada. Bibendum arcu vitae elementum curabitur. Vel risus commodo viverra maecenas accumsan lacus vel. Aliquet nec ullamcorper sit amet risus nullam eget felis. Amet volutpat consequat mauris nunc congue nisi vitae. Ultrices tincidunt arcu non sodales neque sodales. Sed odio morbi quis commodo. Cursus risus at ultrices mi tempus imperdiet. Scelerisque eu ultrices vitae auctor eu augue.\\

    \newpage

    \section{Desarrollo} \label{desarrollo}
\AddToShipoutPictureBG*{\includegraphics[width=\paperwidth,height=\paperheight]{Imagenes/Fondo Capitulo 1.pdf}}
\thispagestyle{plain}

\vspace{0.5cm}

\Large\scshape
\begin{center}
    {\Medium Diseño e implementación de la placa de circuito \\impreso de la plataforma de evaulación}
\end{center}
\normalfont
%\normalsize\vspace{2cm}

\divider

\lipsum[1]\\

\lipsum[2]\\

    \newpage
    \thispagestyle{plain}
    \printbibliography[heading=bibintoc,title={Referencias}]
    \thispagestyle{plain}
    
\end{document}