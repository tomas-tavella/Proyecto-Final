\documentclass[10pt,twoside]{article}           % Twoside para poder poner números en lados alternados

\usepackage[spanish]{babel}                     % Configuración de leguaje
\usepackage[a4paper,top=2cm,bottom=2cm,left=3cm,right=3cm,marginparwidth=1.75cm]{geometry}          % Configurar tamaño de pagina y márgenes
\usepackage{fancyhdr}
\usepackage[autostyle=true]{csquotes}
\usepackage[backend=biber]{biblatex}            % Paquete para bibliografía 
\addbibresource{Bibliografia.bib}              % Llamo al archivo que contiene las referencias
\usepackage{amsmath}
\usepackage{amsfonts}
\usepackage{caption}                            % Para que "Figura x.y" este en bold
\usepackage{graphicx}
%\usepackage[colorlinks=true, allcolors=black]{hyperref}
\usepackage{float}                              % Permite forzar una posición para una figura
%\usepackage{svg}                               % Para poder insertar SVGs
\usepackage[justification=centering,labelsep=space]{caption}   % Para poder centrar captions de figuras
\usepackage{gensymb}                            % Símbolo de grados (\degree)
\usepackage{subfigure}                          % Para poder usar subfiguras (logos de UNLP y FI juntos)
\usepackage{listings}                           % Para poder incluir código
\usepackage{xcolor}                             % Para poder definir colores  
\usepackage{enumitem}                           % Para poder poner letras como items de listas
\usepackage{afterpage}                          % Para poder insertar pagina en blanco
\usepackage[sc]{titlesec}                       % Para cambiar tamaño de headers de secciones, subsecciones, etc.
\usepackage{titletoc}
\usepackage[pagecolor=none]{pagecolor}
\usepackage{siunitx}                            % Para notación científica
\usepackage{mathspec}                           % Para tipografías custom, incluyendo ecuaciones

%-------------------------------------------------------------------------------------------------------%

% Definir todas las tipografías de mathspec

\setmainfont{Montserrat}
\newfontfamily\Thin{Montserrat Thin}
\newfontfamily\ExtraLight{Montserrat ExtraLight}
\newfontfamily\Light{Montserrat Light}
\newfontfamily\Medium{Montserrat Medium}
\newfontfamily\SemiBold{Montserrat SemiBold}
\newfontfamily\Bold{Montserrat Bold}
\newfontfamily\ExtraBold{Montserrat ExtraBold}
\newfontfamily\Black{Montserrat Black}
\setmathrm[BoldFont = {Montserrat SemiBold Italic}]{Montserrat}
\setmathfont(Digits,Latin){Montserrat}

%\setmainfont{Libre Franklin}
%\newfontfamily\Thin{Libre Franklin Thin}
%\newfontfamily\ExtraLight{Libre Franklin ExtraLight}
%\newfontfamily\Light{Libre Franklin Light}
%\newfontfamily\Medium{Libre Franklin Medium}
%\newfontfamily\SemiBold{Libre Franklin SemiBold}
%\newfontfamily\Bold{Libre Franklin Bold}
%\newfontfamily\ExtraBold{Libre Franklin ExtraBold}
%\newfontfamily\Black{Libre Franklin Black}
%\setmathrm{Libre Franklin}
%\setmathfont(Digits,Latin){Libre Franklin}

% Definir formato de titulos de secciones y subsecciones

\titleformat*{\section}{\color{AzulFI_dark}\scshape\huge\centering\Bold}
\titleformat*{\subsection}{\Large\Bold\scshape}
\titleformat*{\subsubsection}{\large\SemiBold}
\titleformat*{\paragraph}{\large\SemiBold}
\titleformat*{\subparagraph}{\large\SemiBold}

% Formato de captions de figuras

\DeclareCaptionFormat{custom}
{%
    \SemiBold\scshape\color{AzulFI_dark} #1#2 \normalfont\normalcolor \textit{#3}                     % #1 es el número, #2 el separador y #3 el texto
}
\captionsetup{format=custom}

% Formato de Table of Contents (ToC)

\contentsmargin{2em}
\dottedcontents{section}[1.5em]{\vspace{0.4cm}\Bold\scshape\color{AzulFI_dark}}{1.5em}{0pc}      % Secciones en Bold en TableofContents
\dottedcontents{subsection}[3.5em]{\vspace{0.1cm}\Medium}{2em}{0.6pc}

% Comando para pagina en blanco sin numero de hoja

\newcommand{\blankpage}
{                          
    \null
    \thispagestyle{empty}
    \addtocounter{page}{-1}
    \newpage
}

\newcommand{\chapterendEven}
{                          
    \null
    \thispagestyle{empty}
    \newpage
}

% Definición de colores

\definecolor{AzulFI}{rgb}{0, 0.394, 0.645}          % 0064A5
\definecolor{AzulFI_dark}{rgb}{0, 0.196, 0.3216}     % 003252
%\definecolor{codegreen}{rgb}{0,0.6,0}
%\definecolor{codegray}{rgb}{0.5,0.5,0.5}
%\definecolor{codepurple}{rgb}{0.58,0,0.82}
%\definecolor{backcolour}{rgb}{0.95,0.95,0.92}

% Estilo del header y footer

\pagestyle{fancy}

\renewcommand{\sectionmark}[1]{\markright{#1}}                          % Borra el numero de sección de \rightmark            
\renewcommand{\subsectionmark}[1]{}                                     % No marca las subsecciones

\fancyfoot[RO,LE]{\normalfont[\ {\color{AzulFI_dark}\SemiBold \thepage}\normalfont\ ]}
\fancyfoot[C]{}
\fancyhead[R]{\color{AzulFI_dark}\scshape{\SemiBold \rightmark}}        % \rightmark da la sección
\fancyhead[L]{\color{AzulFI_dark}\scshape{\SemiBold Capítulo \thesection}}

\fancypagestyle{plain}{
    \fancyhead[L,C,R]{}
    \fancyfoot[C]{}
}

% Parametros para setear como se muestra el codigo

\lstdefinestyle{mystyle}{                       
    backgroundcolor=\color{backcolour},
    commentstyle=\color{codegreen},
    keywordstyle=\color{magenta},
    numberstyle=\tiny\color{codegray},
    stringstyle=\color{codepurple},
    basicstyle=\ttfamily\scriptsize,
    breakatwhitespace=false,         
    breaklines=true,                 
    captionpos=b,                    
    keepspaces=true,                 
    numbers=left,                    
    numbersep=5pt,                  
    showspaces=false,                
    showstringspaces=false,
    showtabs=false,                  
    tabsize=4
}

\lstset{style=mystyle}

% Macro para divisores horizontales

\newcommand{\divider}                           
{
\begin{center}
    \hrulefill
\end{center}
\vspace{0.25cm}
\normalsize
}

% Macro para comillas

\newcommand{\quotes}[1]{``#1''}                 % Comando para double quotes

% Números de figura y ecuación por seccion (1.1 , 2.1, etc.)

\numberwithin{figure}{section}                  % Numeros de figura por seccion
\numberwithin{equation}{section}                % Numeros de ecuacion por seccion

%---------------------------------------------------------------------------------------------------------%
%------------------------------------------ Inicio de Documento ------------------------------------------%
%---------------------------------------------------------------------------------------------------------%

\begin{document}

    \nocite{*}                                  % Se usa para que aparezcan todas la referencias del .bib sin tener que 
                                                % citarlas en el texto
    \newpagecolor{AzulFI}\afterpage{\restorepagecolor\blankpage}    % Página en blanco entre portada y agradecimientos

        \begin{titlepage}
        \begin{center}
            \vspace*{0.5cm}
            \Huge
            \textbf{Diseño y desarrollo de una plataforma experimental de evaluación de sistemas híbridos basados en pilas de combustible}    % Titulo
            \\
            \vspace{0.5cm}
            \huge
            Proyecto Final                                       % Subtitulo
            \\
            \vspace{2cm}
            \Large
            \textbf{Autor:}
            \\
            \large
            \vspace{0.2cm}
            Tomás Tavella - 68371/4
            \\
            \vspace{1cm}
            \Large
            \textbf{Director:}
            \\
            \vspace{0.2cm}
            \large
            Ing. Jorge Anderson Azzano
            \\
            \vspace{0.3cm}
            \Large
            \textbf{Co-director:}
            \\
            \vspace{0.2cm}
            \large
            Dr. Ing. Paul F. Puleston
            \\
            \vfill
            \begin{figure}[H]
                \centering
                \begin{subfigure}
                    \centering
                    \includegraphics[width=0.25\textwidth]{Imagenes/UNLP.pdf}
                \end{subfigure}
                \begin{subfigure}
                    \centering
                    \includegraphics[width=0.32\textwidth]{Imagenes/FI.jpg}
                \end{subfigure}
            \end{figure}
            \vspace{1cm}
            \textit{
            Facultad de Ingeniería
            \\
            Universidad Nacional de La Plata}
            \vspace{1cm}
        \end{center}
    \end{titlepage}
                          
    \newpage 
    \thispagestyle{empty}                       % Para que no se muestre el número de página al final (igual contribuye a la cuenta total)
    \afterpage{\blankpage}

    \Huge
\textbf{Agradecimientos}\\

\normalsize
Lorem ipsum dolor sit amet, consectetur adipiscing elit, sed do eiusmod tempor incididunt ut labore et dolore magna aliqua. Varius quam quisque id diam vel quam. Morbi tristique senectus et netus et malesuada fames ac. Fermentum leo vel orci porta non. Ullamcorper morbi tincidunt ornare massa eget egestas purus viverra accumsan. Non quam lacus suspendisse faucibus. Facilisis volutpat est velit egestas. In mollis nunc sed id semper risus. Lobortis mattis aliquam faucibus purus in massa tempor nec feugiat. Pellentesque elit eget gravida cum sociis natoque penatibus et. Aenean et tortor at risus viverra adipiscing at. Nunc sed blandit libero volutpat. Pretium fusce id velit ut. Sed faucibus turpis in eu.


    \newpage
    \thispagestyle{empty}
    \afterpage{\blankpage}
    \addtocounter{page}{+1}

    \huge
\scshape
\textbf{Resumen}\\

\normalfont\normalsize
Este trabajo consiste del estudio, diseño, implementación y validación de una plataforma experimental para la evaluación de sistemas híbridos de generación energía (SHGE) a partir de pilas o celdas de combustible de tipo PEMFC (\textit{Proton Exchange Membrane Fuel Cell}). Esta plataforma consiste en un sistema de conversión electrónico de tipo CC-CC conmutado y aislado, de topología puente completo; monitoreado mediante la medición de sus estados, y controlado por una excitación de tipo PWM (\textit{Pulse-Width Modulation}) provista por un DSC (\textit{Digital Signal Controller}) de alta performance. Este conversor es requerido para poder adaptar la tensión variable que entrega una celda de combustible a una tensión de salida fija para conectar a un bus común de corriente continua.\\

En el desarrollo de este informe se detallan las tareas realizadas para cumplir este objetivo: el estudio y comprensión de las topologías de conversión CC-CC; la simulación de la topología elegida mediante herramientas de simulación circuitales; el diseño de circuitos auxiliares de excitación, sensado y protección; la implementación del sistema en una placa de circuito impreso mediante software EDA (\textit{Electronic Design Automation}); la programación de los algoritmos de control del sistema; y, finalmente la validación experimental de la plataforma.\\

\vspace{1cm}
\huge
\scshape
\textbf{Abstract}\\

\normalsize\normalfont
This work entails the study, design, implementation and validation of an experimental platform for the evaluation of hybrid energy generation systems based on Proton Exchange Membrane Fuel Cells (PEMFC). This platform incorporates a full-bridge isolated switched-mode DC-DC electronic converter, monitored via the measurement of its state variables, and controlled by a pulse-width modulated (PWM) signal, generated using a high-performance Digital Signal Controller (DSC). This converter provides the adaptation from the variable output voltage of the PEMFC to the fixed voltage of the common DC bus at the system output.\\

This report details the process through which the goals were achieved: study and understanding of the different DC-DC converter topologies, simulation of the selected converter topology using circuit simulation tools, design process of auxiliary circuits, including driver, sensing and protection circuits,  implementation of the system PCB (printed circuit board) through the use of electronic design automation (EDA) software, programming of system control algorithms, and experimental validation of the working platform.\\ 

    \newpage
    \afterpage{\blankpage} 
    \thispagestyle{plain}
    \tableofcontents
    \newpage
    \addtocounter{page}{+1}

    \section{Introducción}

\vspace{0.5cm}

\Large\scshape
Sistema completo en el que se engloba la plataforma de evaluación en estudio. Contexto global como justificación
\normalfont

\divider

Lorem ipsum dolor sit amet, consectetur adipiscing elit, sed do eiusmod tempor incididunt ut labore et dolore magna aliqua. Aliquet enim tortor at auctor urna. Ac orci phasellus egestas tellus rutrum tellus pellentesque eu. Aliquam eleifend mi in nulla. Sit amet cursus sit amet dictum sit amet justo. Tellus orci ac auctor augue mauris augue neque gravida in. Tincidunt dui ut ornare lectus sit amet est. Nulla facilisi morbi tempus iaculis urna id. Quis vel eros donec ac odio tempor orci dapibus. Sed cras ornare arcu dui vivamus. Augue neque gravida in fermentum et. At urna condimentum mattis pellentesque id nibh tortor id. Malesuada fames ac turpis egestas integer eget. Nec feugiat in fermentum posuere urna nec. Pellentesque pulvinar pellentesque habitant morbi. Nunc sed id semper risus in hendrerit gravida.\\

\subsection{Subsección 1}

Parturient montes nascetur ridiculus mus. Pulvinar etiam non quam lacus suspendisse faucibus. Fusce id velit ut tortor pretium viverra suspendisse potenti nullam. Porta non pulvinar neque laoreet suspendisse. Pellentesque id nibh tortor id aliquet lectus. Semper viverra nam libero justo. Vitae tortor condimentum lacinia quis vel eros donec. Ullamcorper velit sed ullamcorper morbi tincidunt. Pellentesque habitant morbi tristique senectus et netus. Non curabitur gravida arcu ac tortor dignissim convallis aenean. Fringilla urna porttitor rhoncus dolor purus non enim praesent. Eget aliquet nibh praesent tristique magna sit amet purus gravida. Orci porta non pulvinar neque. Id porta nibh venenatis cras sed felis. Id neque aliquam vestibulum morbi blandit cursus risus at.\\

\subsection{Subsección 2}

In iaculis nunc sed augue lacus. Odio ut enim blandit volutpat maecenas volutpat. Cras sed felis eget velit aliquet. Risus in hendrerit gravida rutrum quisque non. Risus in hendrerit gravida rutrum quisque non tellus orci. Nec ullamcorper sit amet risus nullam eget felis. Gravida arcu ac tortor dignissim convallis aenean et tortor at. Vehicula ipsum a arcu cursus vitae congue mauris rhoncus. Montes nascetur ridiculus mus mauris vitae ultricies leo integer malesuada. Bibendum arcu vitae elementum curabitur. Vel risus commodo viverra maecenas accumsan lacus vel. Aliquet nec ullamcorper sit amet risus nullam eget felis. Amet volutpat consequat mauris nunc congue nisi vitae. Ultrices tincidunt arcu non sodales neque sodales. Sed odio morbi quis commodo. Cursus risus at ultrices mi tempus imperdiet. Scelerisque eu ultrices vitae auctor eu augue.\\
    
    \newpage

    %\AddToShipoutPictureBG*{\includegraphics[width=\paperwidth,height=\paperheight]{Imagenes/Fondo Capitulo.pdf}}
\section{Análisis de la Plataforma} \label{analisis}
\thispagestyle{plain}

\vspace{0.5cm}

\Large\scshape
\begin{center}
    \textrm{Planteo y estudio de la plataforma de evaluación de celdas de combustible}
\end{center}
\normalfont
%\normalsize

\divider

En este capítulo, se realiza un detallado análisis de la Plataforma Experimental de Evaluación de Módulos de Celdas de Combustible de la figura \ref{diag_plataforma}, la cuál consiste en cuatro subsistemas o bloques distintos: 

\begin{itemize}
    \Medium \item Emulador de Celdas de Combustible
    \Medium \item Convertidor CC-CC Conmutado
    \Medium \item Sistema de Control
    \Medium \item Carga Electrónica Variable\\
\end{itemize}

\begin{figure}[h]
    \centering
    \includegraphics[scale=0.4]{Imagenes/Plataforma Experimental.pdf}
    \caption{Diagrama de la plataforma experimental de evaluación, con sus cuatro bloques principales.}
    \label{diag_plataforma}
\end{figure}

Esta plataforma, con sus distintos bloques, se encarga de evaluar la \textit{performance} de celdas de combustible conectadas a un sistema híbrido de generación. Con este fin, un emulador de celdas de combustible toma el puesto de celdas de combustible reales, y una carga electrónica variable se utiliza para simular cualquier tipo de condiciones de carga que se deseen en el bus de CC. Para poder conectar el emulador a la carga, se debe implementar un subsistema (Convertidor CC-CC Conmutado) que adapte los niveles de tensión de salida del emulador de celdas a la tensión fija de salida en la carga, adicionando un módulo de control que monitorea los estados del convertidor, y los controla mediante los disparos de las llaves del puente completo.\\

El principal objetivo de este proyecto es el diseño e implementación de la etapa de adaptación de tensión (es decir, el convertidor con su sistema de control), pero se hace un estudio detallado de todas los componentes de la plataforma, de manera de obtener un entendimiento más completo de todo el sistema. Por esta razón, a continuación se hace un análisis en profundidad de cada una de las partes individuales, comenzando por el emulador de celdas de combustible.\\

\subsection{Celdas de Combustible}

A pesar de que las celdas de combustible son una tecnología de hace más de un siglo y medio (desarrollada por primera vez por el físico galés Sir William Grove en 1842), hoy en día despiertan un particular interés en el campo de la generación renovable por su alta eficiencia, su dependencia en recursos obtenibles fácilmente de maneras ambientalmente amigables, y por la generación de agua como único deshecho.\\

Por estas razones se eligió trabajar con esta tecnología, particularmente con el tipo de celda más común hoy en día, las Celdas de Combustible de Membrana de Intercambio Protónico o PEMFC (del inglés \textit{Proton Exchange Membrane Fuel Cell}), cuyo funcionamiento se profundiza más adelante.\\

\subsubsection{Principio de Funcionamiento de las Celdas de Combustible}

Esencialmente, una celda de combustible es una celda galvánica o celda voltáica en la cual la energía libre de una reacción química redox (entre un combustible y un agente oxidante o \textit{comburente}) se convierte a energía eléctrica mediante una corriente y una diferencia de potencial$^{[FC-FundamentalsAndApplications]}$. En nuestro caso particular, el combustible es el hidrógeno molecular ($H_2$), el agente oxidante es el oxígeno ($O_2$) abundante en la atmósfera, y los productos son la energía eléctrica y el agua ($H_2O$) según indica la siguiente ecuación química balanceada.

\begin{equation}\label{redox_celda}
    2H_2\ +\ O_2\ \longrightarrow\ 2H_2O
\end{equation}

La estructura interna de una celda de combustible, visible en la figura \ref{fuel_cell}, consiste de un ánodo (electrodo negativo) al cual ingresan las moléculas de hidrógeno, un cátodo (electrodo positivo) en el que ingresa el oxígeno y se despide el agua, y un electrolito como como interfaz entre ánodo y cátodo. La carga es conectada entre el ánodo y el cátodo.

\begin{figure}[h]
    \centering
    \includegraphics[scale=0.35]{Imagenes/Fuel Cell.png}
    \caption{Esquema de una celda de combustible, con todos sus componentes indicados (Placeholder).}
    \label{fuel_cell}
\end{figure}

La reacción redox de la ecuación \ref{redox_celda}, dentro de una celda de combustible como la del esquema, en realidad se separa en dos reacciones parciales distintas:

\begin{equation}\label{redox_anodo}
    2H_2\ \longrightarrow\ 4H^{+}\ +\ 4e^-
\end{equation}

\begin{equation}\label{redox_catodo}
    4H^{+}\ +\ 4e^-\ +\ O_2\longrightarrow\ 2H_2O
\end{equation}

De esta manera, alimentado simultáneamente el terminal negativo con combustible (hidrógeno) y el terminal positivo con oxidante (oxígeno) se producen las dos reacciones en las superficies de contacto del electrolito:

\begin{itemize}
    \item \textbf{En el ánodo} las moléculas de $H_2$ pierden sus electrones, bifurcándose los iones positivos de hidrógeno ($H^{+}$) por el electrolito y los electrones libres a través de la carga (ecuación \ref{redox_anodo}). Es una reacción exotérmica (libera calor) que resulta en el calentamiento de la celda.
    \item \textbf{En el cátodo} los iones $H^{+}$ del electrolito, los electrones libres, y las moléculas de oxígeno reaccionan para formar como producto el agua (ecuación \ref{redox_catodo}).
\end{itemize}

Mediante este proceso electroquímico se generan dos corrientes distintas: una corriente interna de iones $H^{+}$ (cargas positivas) en el electrolito, desde el ánodo hacia el cátodo; y una corriente externa de electrones $e^-$ (cargas negativas) circulando por la carga, en el mismo sentido que la corriente de iones. Esta última corriente de electrones es la que nos resulta útil para poder alimentar algún tipo de carga.\\

\subsubsection{De Celda a Pila de Combustible}

Sin embargo, una celda de combustible individual como en la figura \ref{fuel_cell} no es capaz de entregar una diferencia de potencial lo suficientemente alta para la gran mayoría de las aplicaciones, con una tensión de celda común situada entre 0.7 V y 1.3 V, dependiendo de varios aspectos constructivos específicos de la celda.\\

Entonces, para obtener un dispositivo con una tensión de salida de niveles utilizables, esta tecnología generalmente se comercializa en forma de pilas o \textit{stacks} de celdas individuales conectadas en serie como se ve en la figura \ref{fuel_cell_stack}, generalmente de entre diez y cien celdas, cuya tensión es la suma de la tensión de cada celda que la compone.\\

Esto se logra, como dice su nombre, apilando todas las celdas de combustible para formar el \textit{stack}, utilizando placas de interconexión para conectar electrodos de polaridad opuesta de dos celdas aledañas (es decir, se conecta el ánodo de una celda con el cátodo de la siguiente); además de cumplir la función de aislar el combustible de una celda del agente oxidante de la celda contigua. Este es el tipo de conexionado de celdas más común, llamado \textit{Planar-Bipolar Stacking} o Apilado Planar-Bipolar$^{[FCHandbook]}$.

\begin{figure}[H]
    \centering
    \includegraphics[scale=0.2]{Imagenes/Fuel Cell.png}
    \caption{Figura de un stack de celdas.}
    \label{fuel_cell_stack}
\end{figure}

\subsubsection{Aspectos Constructivos de Celdas}

Habiendo repasado el principio básico de funcionamiento de las celdas de combustible, ahora se realizará una breve descripción de los aspectos constructivos de las mismas. La utilización de distintos materiales y composiciones de las partes que las componen derivan en distintos tipos de celdas, que, a pesar de funcionar bajo el mismo principio básico, poseen cada una sus ventajas y desventajas que las hacen más o menos apropiadas para distintas aplicaciones.\\

Como las reacciones químicas ocurren en superficies microscópicas dónde alguno de los electrodos está en contacto con el electrolito, generalmente los electrodos se fabrican de materiales porosos que aumentan la posible superficie de contacto entre ambas fases, acelerando las reacciones necesarias para producir energía. Sin embargo, en muchos casos, a temperaturas bajas los materiales de los electrodos no son capaces de producir la suficiente actividad electroquímica, por lo que suelen agregarse pequeñas cantidades de catalizador en las zonas de contacto para acelerar la reacción.\\

En tanto al electrolito, estos suelen estar hechos de materiales en estado líquido o sólido, dependiendo del tipo de celda, pero siempre deben tener una alta conductividad de iones positivos, de manera que los iones $H^{+}$ circulen solo por el elctrolito y no por el circuito externo. Adicionalmente, este material debe actuar de barrera física para evitar que se mezclen los flujos de combustible y comburente.\\

En tanto a la geometría de las celdas, se ha expermientado con una gran variedad de formas para los electrodos y electrolitos pero, hoy en día las pilas que se producen son mayormente planas, y en algunos casos tubulares.\\

\subsubsection{Tipos de Celdas}

Hay muchas formas de clasificar las distintas tecnologías de celdas, pero en nuestro caso nos vamos a enfocar en la distinción más común, que es la clasificación según el material usado como electrolito. Hoy en día, hay seis tipos distintos de celdas segun electrolito, descritas a continuación, con una mayor profundización mayor en las del tipo PEMFC que se mencionaron anteriormente, ya que son este tipo de pilas las que nos interesa en nuestra aplicación particular.

\begin{itemize}
    \item \textbf{Celda de Combustible Alcalina (AFC)}\\
    Hola
    \item \textbf{Celda de Combustible de Metanol Directo (DMFC)}\\
    Hola
    \item \textbf{Celda de Combustible de Ácido Fosfórico (PAFC)}\\
    Hola
    \item \textbf{Celda de Combustible de Carbonato Fundido (MCFC)}\\
    Hola
    \item \textbf{Celda de Combustible de Óxido Sólido (SOFC)}\\
    Hola
    \item \textbf{Celda de Combustible de Membrana de Intercambio Protónico (PEMFC)}\\
    Hola
\end{itemize}

\newpage

\subsection{Convertidor CC-CC Conmutado}

Un convertidor CC-CC es un dispositivo electrónico que tiene como objetivo convertir una tensión continua, generalmente no regulada (es decir que no es fija), $V_{in}$ a la entrada, a una tensión continua regulada $V_{out}$ de distinta magnitud a la salida, transfiriendo la mayor cantidad de energía posible de la entrada hacia la salida. Dependiendo del tipo de convertidor, esta tensión de salida puede ser menor, mayor o tanto menor como mayor a la tensión de entrada.\\

Estos convertidores son de interés para nuestra aplicación, ya que la tensión $V_{stack}$ que entrega la pila (ecuación \ref{v_stack}) es una tensión continua no regulada, que varía apreciablemente con la corriente demandada; mientras que a la salida se demanda una tensión fija y regulada $V_{bus}$ para conectar al bus de continua del sistema híbrido de la figura \ref{SHGE}.\\

La forma más básica que se podría concebir para un dispositivo que cumpla esta función es la de un simple divisor resistivo, en el cual la tensión $V_{out}$ depende de las resistencias $R_1$ y $R_2$ y la tensión de entrada $V_{in}$.

\begin{equation*}
    V_{out} = V_{in}\cdot\frac{R_2}{R_1+R_2}
\end{equation*}

Entonces, variando la relación entre $R_1$ y $R_2$, se puede variar la tensión $V_{out}$ entre tensión nula y $V_{in}$. Sin embargo, se necesita solo un análisis superficial de esta topología para ver que no es viable para ningún tipo de aplicación, más que nada por su pobre eficiencia energética (para obtener una tensión igual a la mitad de la entrada, se pierde la mitad de la potencia en disipación resistiva).\\

Los convertidores CC-CC se suelen separar en dos principales categorías: los {\Medium reguladores lineales}, que son un caso complejizado del divisor resistivo donde se utiliza un transistor como resistencia variable (además de un diodo para regular la tensión de salida); y los {\Medium convertidores conmutados}, en los cuales uno o más transistores, actuando como llaves, son conmutados a alta frecuencia y junto con dispositivos que almacenan energía (como inductores y capacitores) producen una tensión continua a la salida.\\

Dado que para esta plataforma se utiliza un convertidor conmutado (principalmente por su gran ventaja en eficiencia energética), se enfocará el análisis únicamente en éstos; comenzando por una explicación de los conceptos básicos necesarios para comprender su funcionamiento.\\

\subsubsection{Conceptos Básicos}

Como se detalló más arriba, los convertidores CC-CC conmutados consisten, en su forma más básica, en una fuente de continua no regulada a la entrada; y un transistor (que puede ser BJT, MOSFET o IGBT) que, mediante una excitación en su tercer terminal, se conmuta entre los modos de alta impedancia e impedancia nula, actuando como llave abierta y llave cerrada respectivamente. La proporción del tiempo total de ciclo ($T_s$) en la que el transistor está conduciendo ($t_{on}$) se denomina {\Medium ciclo de trabajo} o {\Medium \textit{duty cycle}} y se suele simbolizar con la {\Medium letra \textit{D}}. Como se verá más adelante, este es un parámetro crucial para el funcionamiento de este tipo de convertidores, ya que controlándolo se puede variar el nivel de tensión y corriente de salida.

\begin{figure}[H]
    \centering
    \includegraphics[scale=0.5]{Imagenes/Duty Cycle.pdf}
    \caption{Una forma de onda cuadrada con ciclo de trabajo $D$ del 25 \%.}
    \label{dutycycle}
\end{figure}

Los convertidores CC-CC conmutados se clasifican en dos grandes grupos, usando como criterio la existencia de aislación galvánica entre la entrada no regulada y la salida regulada:

\begin{itemize}
    \item {\SemiBold Convertidores No Aislados:} son los convertidores que no tienen aislación galvánica entre entrada y salida, como por ejemplo los convertidores reductores y elevadores (\textit{buck} y \textit{boost}), y por lo tanto son los mas simples de los dos tipos.
    \item {\SemiBold Convertidores Aislados:} son los convertidores que tienen su entrada y salida aisladas galvánicamente por medio de un transformador de alta frecuencia, por ejemplo los de tipo \textit{flyback} y \textit{forward}. El convertidor de esta plataforma, de tipo puente completo, cae dentro de esta categoría.
\end{itemize}

En la siguiente sección se va a detallar el funcionamiento de los dos convertidores no aislados más sencillos, los convertidores reductor y elevador, a manera de introducir los principios de funcionamiento de convertidores conmutados que van a ser necesarios para luego poder entender las topologías más complejas que se utilizan en esta plataforma.\\

\subsubsection{El Convertidor Reductor}

La forma más básica posible de un convertidor conmutado tiene un esquema circuital similar al convertidor lineal mencionado más arriba, con la diferencia de que el transistor, (que previamente actuaba como una resistencia variable para conformar el divisor resistivo) en este caso, actúa como el interruptor del circuito, conmutando entre llave abierta y cerrada (figura \ref{proto_reductor}). Para este análisis vamos a considerar que el dispositivo semiconductor actúa como una llave ideal, sin impedancia cuando está cerrado y con impedancia infinita cuando está abierto.

\begin{figure}[H]
    \centering
    \includegraphics[scale=0.8]{Imagenes/Proto Reductor.pdf}
    \caption{Circuito de un convertidor conmutado básico, y su equivalente con el transistor $Q_1$ como llave ideal.}
    \label{proto_reductor}
\end{figure}

Entonces, si se aplica una señal de control como la de la figura \ref{dutycycle} al interruptor, durante un período $T_s$ de la señal ocurren dos cosas distintas:

\begin{itemize}
    \item {\SemiBold Durante el tiempo $\mathbf{t_{on}}$}, el transistor se comporta como una llave cerrada y permite la libre circulación de corriente. Entonces, esta corriente circula por la carga $R_L$, donde, por la Ley de Ohm, cae una tensión igual a la tensión de entrada, es decir, que la tensión de salida {\Medium \textit{v\textsubscript{o}} es igual a la tensión de entrada \textit{V\textsubscript{s}}}.
    \item {\SemiBold Durante el tiempo $\mathbf{t_{off}}$}, el transistor pasa a comportarse como una llave abierta, por lo que restringe completamente la circulación de corriente. Por lo tanto, la caída de tensión en la carga $R_L$ es nula, es decir, que la tensión de salida {\Medium \textit{v\textsubscript{o}} es nula}.
\end{itemize}

Uniendo estos dos comportamientos, se puede ver que la forma de la tensión de salida es análoga a la forma de onda cuadrada que controla al interruptor (de la figura \ref{dutycycle}), oscilando entre \SI{0}{\volt} y $V_s$.

\begin{equation}\label{valor_medio_reductor}
    \bar{v}_o = \frac{1}{T_s}\int\limits^{T_s}_0 v_0(t) dt = \frac{1}{T_s}\int\limits^{DT_s}_0 V_s dt = V_s\cdot D \leq V_s
\end{equation}

Calculando el valor medio de $v_o$ en la ecuación \ref{valor_medio_reductor}, este resulta ser directamente proporcional al ciclo de trabajo de la señal de control, variando entre \SI[]{0}[]{\volt} y la tensión de entrada $V_s$, para ciclos de trabajo entre \num{0} y \num{1} respectivamente. Es decir, la {\Medium tensión media de salida es menor o igual a la de entrada} (esto se puede ver sin necesidad de cálculo, ya que si la salida es igual a la entrada por una proporción del tiempo total, su valor medio necesariamente debe ser menor, o como mucho igual, al valor de la entrada) y se controla directamente con la variación de $D$.\\

En principio, si se considera el transistor como interruptor ideal, la eficiencia de este dispositivo es del 100 \%, ya que durante el tiempo $t_{off}$ no circula ninguna corriente (por lo tanto no hay disipación de ningún tipo), y durante $t_{on}$ no hay caída de tensión en el transistor. En la realidad, los transistores no actúan como llaves ideales, si no que tienen ciertas no idealidades que resultan en pérdidas de energía: no tienen impedancia perfectamente nula como llave cerrada, ni impedancia infinita como llave abierta, además de poseer pérdidas a la hora de conmutar.\\

Sin embargo, en muchos casos y aplicaciones (incluido el de este trabajo) no es suficiente obtener una salida de pulsos y controlar su tensión media, si no que se necesita obtener una tensión puramente continua directamente en la salida, como puede ser el caso para una fuente de alimentación.\\

Para solucionar este problema, se agrega un filtro pasa-bajos LC a la salida luego del interruptor, que se encarga de eliminar los componentes de alta frecuencia relacionados a la conmutación, dejando pasar únicamente los componentes de continua. El convertidor que resulta es la topología de convertidor CC-CC conmutado más sencilla: el {\Medium convertidor reductor} o {\Medium \textit{buck}} de la figura \ref{reductor}, que obtiene su nombre porque, como se ve en la ecuación \ref{valor_medio_reductor}, reduce la tensión de entrada.\\$\bar{i}_C$

\begin{figure}[H]
    \centering
    \includegraphics[scale=0.8]{Imagenes/Reductor.pdf}
    \caption{Circuito de un convertidor reductor o buck, con componentes ideales.}
    \label{reductor}
\end{figure}

Además del filtro ya mencionado, se agrega un diodo de rueda libre o \textit{flyback} en derivación entre el transistor y el inductor (diodo $D_1$ de la figura \ref{reductor}). Este dispositivo cumple la función de proveer un camino de circulación para la corriente $i_L$ del inductor cuando el interruptor se encuentra abierto, que resulta necesario ya que la corriente sobre un inductor no puede variar abruptamente. Entonces, cuando el interruptor está abierto, el diodo entra en polarización directa y permite la circulación de corriente; mientras que cuando está el interruptor cerrado, el diodo se polariza con una tensión inversa $V_s$ y actúa como un circuito abierto, eliminando su influencia sobre el convertidor durante $t_{on}$.\\

Durante su funcionamiento en estado estacionario, los convertidores reductores (y todos los convertidores CC-CC) tienen las siguientes propiedades:

\begin{itemize}
    \item La corriente $i_L$ sobre el inductor es periódica de período $T_s$, es decir, $i_L(t+T_s) = i_L(t)$.
    \item La tensión media $\bar{V}_L$ que cae en el inductor es nula, ya que si no lo fuera su corriente crecería sin límite.
    \item La corriente media $\bar{i}_C$ que circula por el capacitor es nula, ya que si no lo fuera su tensión crecería sin límite.
    \item La potencia absorbida por la carga es igual a la potencia entregada por la fuente. Para componentes no ideales, las pérdidas son entregadas por la fuente de entrada.
\end{itemize}

\paragraph{Análisis Detallado}

Ahora se va a realizar un análisis más en profundidad de la topología. Pero antes, es necesario aclarar el conjunto de condiciones que se asumirán, necesarias para simplificar y facilitar la comprensión de esta explicación:

\begin{enumerate}
    \item El circuito opera en estado estacionario, es decir que todos las respuestas transitorias ya se extinguieron.
    \item La corriente del inductor es continua, es decir que circula siempre en la misma dirección.
    \item El capacitor $C$ es lo suficientemente grande como para mantener la tensión de salida constante.
    \item El período de conmutación es $T_s$, con $t_{on} = DT_s$ y $t_{off} = (1-D)T_s$.
    \item Todos los componentes son ideales.
\end{enumerate}

Para poder determinar la tensión de salida $v_o$ del sistema, se va a determinar primero la corriente y tensión del inductor $L$ del filtro de salida, para cada uno de los dos estados del circuito: {\Medium llave abierta} y {\Medium llave cerrada}. Para cumplir la condición de funcionamiento en estado estacionario, la corriente $i_L$ debe tener una variación total nula durante un período $T_s$ (es decir que la corriente debe ser la misma al principio y final de un ciclo), y, como se mencionó más arriba, su tensión media debe ser idénticamente nula.\\

\subparagraph{Llave Cerrada}

Al estar la llave cerrada durante el tiempo $t_{on} = DT_s$, la tensión de entrada $V_s$ cae directamente sobre el diodo $D_1$, polarizándolo con una tensión inversa que no permite que circule corriente por el mismo, y en consecuencia, neutralizando su efecto sobre el circuito. Se puede ver el circuito equivalente para este estado en la figura \ref{reductor_llave_cerrada}.

\begin{figure}[H]
    \centering
    \includegraphics[scale=0.8]{Imagenes/Reductor Llave Cerrada.pdf}
    \caption{Circuito equivalente de un convertidor reductor para llave cerrada.}
    \label{reductor_llave_cerrada}
\end{figure}

Entonces, recordando que la tensión que cae sobre una bobina es proporcional a la corriente que circula sobre ella (con $L$ como constante de proporcionalidad), la tensión sobre el inductor del circuito resulta

\begin{equation}\label{ec_tensionL_cerrada}
    v_L = V_s - v_o = L\frac{di_L}{dt}
\end{equation}

Como la tensión, y por lo tanto la derivada de la corriente, son valores constantes y positivos, la corriente por el inductor es descrita por una recta de pendiente positiva. El cambio neto de corriente $(\Delta i_L)_{cerrado}$ mientras la llave permanece cerrada es entonces

\begin{equation*}
    \frac{di_L}{dt} = \frac{(\Delta i_L)_{cerrado}}{\Delta t} = \frac{(\Delta i_L)_{cerrado}}{DT_s} = \frac{V_s - v_o}{L}\\
\end{equation*}

Reorganizando:

\begin{equation}\label{deltaiL_cerrada}
    \boxed{
        (\Delta i_L)_{cerrado} = \left(\frac{V_s - v_o}{L}\right)DT_s
    }
\end{equation}

\subparagraph{Llave abierta}

Ahora, al abrirse la llave durante el tiempo $t_{off} = (1-D)T_s$, el diodo entra en modo de polarización directa, permitiendo la circulación de la corriente acumulada en el inductor. La fuente queda desconectada y no entrega energía, conformándose el circuito equivalente de la figura \ref{reductor_llave_abierta}.

\begin{figure}[H]
    \centering
    \includegraphics[scale=0.8]{Imagenes/Reductor Llave Abierta.pdf}
    \caption{Circuito equivalente de un convertidor reductor para llave abierta.}
    \label{reductor_llave_abierta}
\end{figure}

En este intervalo de tiempo, la tensión sobre el inductor es

\begin{equation}\label{ec_tensionL_abierta}
    v_L = -v_o = L\frac{di_L}{dt}
\end{equation}

Entonces, aplicando un razonamiento análogo al del período de llave cerrada, con la diferencia que en este caso, al ser la tensión $v_L$ negativa, la recta de la corriente $i_L$ es decreciente, se obtiene que el cambio neto de corriente $(\Delta i_L)_{abierto}$ mientras la llave está abierta es

\begin{equation*}
    \frac{di_L}{dt} = \frac{(\Delta i_L)_{abierto}}{(1-D)T_s} = \frac{-v_o}{L}\\
\end{equation*}

Reordenando:

\begin{equation}\label{deltaiL_abierta}
    \boxed{
        (\Delta i_L)_{abierto} = -\left(\frac{v_o}{L}\right)(1-D)T_s
    }
\end{equation}

Como se mencionó antes, para este análisis se asumió el funcionamiento en estado estacionario, por lo que la suma de los cambios netos de corriente de las ecuaciones \ref{deltaiL_cerrada} y \ref{deltaiL_abierta} para ambos estados del circuito debe ser igual a cero.

\begin{equation}
    (\Delta i_L)_{cerrado} + (\Delta i_L)_{abierto} = 0
\end{equation}

Reemplazando ambas variables por sus expresiones, se obtiene

\begin{equation*}
    \left(\frac{V_s - v_o}{L}\right)DT_s - \left(\frac{v_o}{L}\right)(1-D)T_s  = 0
\end{equation*}

Despejando de la ecuación anterior, se consigue una expresión para la tensión de salida $v_o$ de este convertidor.

\begin{equation}\label{vo_reductor}
    \boxed{
        v_o = V_sD
    }
\end{equation}

Este resultado es idéntico al de la ecuación \ref{valor_medio_reductor} obtenido para el convertidor básico de la figura \ref{proto_reductor}. En conclusión, {\Medium en un convertidor reductor, la tensión de salida siempre es menor o igual a la entrada}.\\

Evidentemente, por el resultado obtenido en la ecuación \ref{vo_reductor}, la salida se controla únicamente con el ciclo de trabajo $D$ del transistor. Por ejemplo, si aumenta la tensión de alimentación $V_s$ pero se desea mantener $v_o$ a un nivel constante, se compensa este aumento con una disminución del ciclo de trabajo (o viceversa). Si se agrega un sensor que mida la tensión de salida, se puede implementar un lazo de control automático que mantiente $v_o$ fijada a una referencia mediante la variación de $D$.\\

\subsubsection{Convertidores CC-CC Aislados}

Habiendo entendido el funcionamiento del convertidor reductor en la anterior sección (que cae en la categoría de convertidores CC-CC no aislados), ahora vamos a pasar a los convertidores CC-CC aislados, categoría en la cual se encuentra el convertidor tipo puente completo de esta plataforma.\\

Los convertidores aislados son generalmente utilizados en fuentes de alimentación de corriente continua, y a diferencia de los no aislados, tienen un transformador de alta frecuencia de por medio, para generar una {\Medium aislación galvánica entre la entrada y la salida}. Además, como los transformadores solo conducen corriente alterna, a su salida debe incluirse algún tipo de circuito rectificador para transformarla a corriente continua para alimentar a la carga.\\

Es claro que el adicionado de un transformador agrega una mayor complejidad al circuito. Entonces, ¿por qué se busca esta aislación galvánica? Sin la aislación interpuesta, nuestra salida va a compartir la conexión a tierra con la fuente de alimentación, (que suelen tener tierras muy ruidosas) introduciendo ruido no deseado a la salida. En muchas aplicaciones hay una gran sensibilidad al ruido en la carga, por lo que es deseable mantenerlo lo más bajo posible, incluso si agrega complejidad al diseño. Adicionalmente, la presencia de aislación galvánica presenta una ventaja en cuestiones de seguridad, tanto para proteger a quienes operan con el circuito como para protección de los componentes del mismo circuito.\\

Otra ventaja es la mayor flexibilidad que un transformador en la etapa de continua aporta al diseño, ya que variando la relación de vueltas entre bobinados (por ejemplo con el uso de múltiples bobinados) se puede variar la tensión de salida entre distintos niveles.\\

Ahora se procederá a derivar las distintas topologías de convertidores aislados, partiendo del convertidor reductor (no aislado) que se explico más arriba. Estos convertidores que obtendremos los vamos a llamar {\Medium convertidores aislados derivados del reductor} o \textit{isolated buck-derived converters}\textsuperscript{\cite{SoftSwitchPWM}}, comenzando por el convertidor \textit{forward}.\\

\paragraph{El Convertidor Forward}

Si tomamos el circuito del reductor de la figura \ref{reductor}, y le agregamos un transformador de alta frecuencia entre la llave $Q_1$ y el diodo $D_1$, se obtiene la aislación galvánica buscada, como se observa en el circuito de la figura \ref{desarrollo_forward1}.

\begin{figure}[H]
    \centering
    \includegraphics[scale=0.8]{Imagenes/Desarrollo Forward 1.pdf}
    \caption{Convertidor reductor con un transformador interpuesto entre la llave $Q_1$ y el diodo $D_1$.}
    \label{desarrollo_forward1}
\end{figure}

Cuando la llave está cerrada, la tensión $V_s$ de entrada se aplica sobre el bobinado primario del transformador, traduciéndose a una tensión de la misma polaridad (por la ubicación de los puntos homólogos) pero afectada por la relación de vueltas $n$. Esto genera que el núcleo ferromagnético del transformador se magnetice, y aumente su flujo de magentización $\phi_m$.\\

Cuando la llave se abre, la corriente del inductor de filtro circula por el diodo $D_1$, cortocircuitando el bobinado secundario del transformador. Esto fuerza que la tensión y la corriente del transformador se anulen, y por lo tanto, el flujo magnetizante se mantiene constante.\\

Entonces, durante un período de conmutación $T_s$, el flujo $\phi_m$ del núcleo del transformador tiene un incremento neto. Pasados suficientes períodos, este flujo aumenta lo suficiente como para saturar el transformador, cosa que no es deseable, ya que puede resultar en corrientes elevadas y eventualmente, la destrucción del transistor de potencia que actúa como llave.\\

Para solucionar este problema, se debe agregar algún circuito auxiliar de restablecimiento del núcleo que, durante el período en el que la llave está abierta, aplique una tensión negativa en el bobinado primario y permita una circulación inversa de corriente para restablecer el flujo magnetizante a su valor original.\\

Pero, al aplicar esta tensión negativa en el primario, se refleja en una tensión negativa del secundario que polariza en directa a $D_1$, cortocircuitando este bobinado. Para arreglar este inconveniente, se puede agregar un diodo rectificador $D_R$ en serie con el bobinado secundario, que no permita la circulación inversa de corriente.\\

Teniendo esto en cuenta, se ve en la figura \ref{forward} el circuito de un {\Medium convertidor \textit{forward}} derivado de un reducto, dónde se agregaron el circuito de restablecimiento de núcleo, compuesto por un bobinado auxiliar y un diodo $D_r$ en serie; en paralelo con el bobinado primario y la llave $Q_1$ (en posiciones invertidas); y el diodo rectificador $D_R$ en el secundario (respecto a la figura \ref{desarrollo_forward1}).\\

\begin{figure}[H]
    \centering
    \includegraphics[scale=0.8]{Imagenes/Desarrollo Forward 1.pdf}
    \caption{Circuito de un convertidor forward, desarrollado a partir del circuito de un reductor (Placeholder).}
    \label{forward}
\end{figure}

Dado que este circuito es similar a un convertidor reductor, solo que con un transformador de relación de vueltas $n$ interpuesto (y los circuitos auxiliares que no afectan la salida), se puede ver que su relación entrada-salida será similar a la del convertidor en el que se basa (ecuación \ref{vo_reductor}) pero afectada por la relación de vueltas del transformador.\textsuperscript{\cite{PotenciaHart}}

\begin{equation}\label{vo_forward}
    \boxed{
        v_o = \left(\frac{N_2}{N_1}\right)V_sD = nV_sD
    }
\end{equation}

Con estos resultados se puede ver la flexibilidad aportada por el transformador: a pesar de ser muy similar a un circuito que solo permite reducir la tensión, con la relación de vueltas se puede obtener cualquier nivel de tensión que se desee a la salida. Sin embargo, al tener que restablecer la magnetización del núcleo, se suele limitar el ciclo de trabajo a 50 \% para poder lograr la demagnetización completa.\textsuperscript{\cite{SoftSwitchPWM}}\\

Otra consecuencia, que surge del agregado del bobinado auxiliar, es la tensión que el transistor debe soportar cuando se encuentra abierto. Si las vueltas de ambos bobinados (primario y auxiliar) son iguales, en el transistor abierto cae una tensión $2V_s$, es decir el doble de la tensión de entrada. Esto puede ser problemático, ya que se deben conseguir transistores que soporten una mayor tensión, característica que viene de mano con un desempeño de alta frecuencia disminuido y un mayor precio.\\

[Alguna cosa para conectar el forward con el push-pull, como lo que dice en el primer párrafo de push-pull del libro]. Esto se va a desarrollar partiendo del circuito de la figura \ref{forward}.\\

\paragraph{El Convertidor Push-Pull}

\begin{center}
    {\Thin Thin} {\ExtraLight ExtraLight} {\Light Light} Regular {\Medium Medium}  {\SemiBold SemiBold} {\Bold Bold}
\end{center}

\newpage

\subsection{Sistema de Control}

{\Bold\scshape Falta completar esta sección.}\\

\lipsum[1]\\

\lipsum[2]\\

\lipsum[3]\\

\newpage

\subsection{Carga Electrónica Variable}

{\Medium ITECH IT8514B+} - \SI[]{500}[]{\volt}/\SI[]{60}[]{\ampere}/\SI[]{1500}[]{\watt}\\

{\Bold\scshape Falta completar esta sección.}\\

\lipsum[1]\\

\lipsum[2]\\

\lipsum[3]\\

\afterpage{\blankpage}
\newpage

    \newpage

    %\printbibliography 
    
\end{document}