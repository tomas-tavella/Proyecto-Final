\documentclass{article}

%------------------------------------------ Paquetes útiles ------------------------------------------%

\usepackage[spanish]{babel}                     % Configuración de leguaje
\usepackage[a4paper,top=2cm,bottom=2cm,left=3cm,right=3cm,marginparwidth=1.75cm]{geometry}          % Configurar tamaño de pagina y márgenes
\usepackage[autostyle=true]{csquotes}
\usepackage[backend=biber]{biblatex}            % Paquete para bibliografía
\usepackage{amsmath}
\usepackage{amsfonts}
\usepackage{caption}                            % Para que "Figura x.y" este en bold
\usepackage{graphicx}
%\usepackage[colorlinks=true, allcolors=black]{hyperref}
\usepackage{float}                              % Permite forzar una posición para una figura
%\usepackage{svg}                               % Para poder insertar SVGs
\usepackage[justification=centering]{caption}   % Para poder centrar captions de figuras
\usepackage{gensymb}                            % Símbolo de grados (\degree)
\usepackage{subfigure}                          % Para poder usar subfiguras (logos de UNLP y FI juntos)
\usepackage{listings}                           % Para poder incluir código
\usepackage{xcolor}                             % Para poder definir colores  
\usepackage{enumitem}                           % Para poder poner letras como items de listas
\usepackage{afterpage}                          % Para poder insertar pagina en blanco
\usepackage[sc]{titlesec}                       % Para cambiar tamaño de headers de secciones, subsecciones, etc.
\usepackage{titletoc}
\usepackage[pagecolor=none]{pagecolor}
\usepackage{siunitx}                            % Para notación científica
\usepackage{mathspec}                           % Para tipografías custom, incluyendo ecuaciones

%-------------------------------------------------------------------------------------------------------%

\addbibresource{Bibliografia.bib}               % Llamo al archivo que contiene las referencias

% Definir todas las tipografías de mathspec

\setmainfont{Montserrat}
\newfontfamily\Thin{Montserrat Thin}
\newfontfamily\ExtraLight{Montserrat ExtraLight}
\newfontfamily\Light{Montserrat Light}
\newfontfamily\Medium{Montserrat Medium}
\newfontfamily\SemiBold{Montserrat SemiBold}
\newfontfamily\Bold{Montserrat Bold}
\newfontfamily\ExtraBold{Montserrat ExtraBold}
\newfontfamily\Black{Montserrat Black}
\setmathrm{Montserrat}
\setmathfont(Digits,Latin){Montserrat}

% Definir formato de titulos de secciones y subsecciones

\titleformat*{\section}{\scshape\huge\centering\Bold}
\titleformat*{\subsection}{\Large\Bold}
\titleformat*{\subsubsection}{\large\SemiBold}
\titleformat*{\paragraph}{\large\SemiBold}
\titleformat*{\subparagraph}{\large\SemiBold}

% Formato de captions de figuras

\DeclareCaptionFormat{custom}
{%
    \SemiBold #1#2 \normalfont #3                       % #1 es el número, #2 el separador y #3 el texto
}
\captionsetup{format=custom}

\dottedcontents{section}[1.5em]{\Bold}{1.5em}{0pc}      % Secciones en Bold en TableofContents

% Comando para pagina en blanco sin numero de hoja

\newcommand{\blankpage}
{                          
    \null
    \thispagestyle{empty}
    \addtocounter{page}{-1}
    \newpage
}

% Definición de colores

\definecolor{AzulFI}{rgb}{0, 0.394, 0.645}
%\definecolor{codegreen}{rgb}{0,0.6,0}
%\definecolor{codegray}{rgb}{0.5,0.5,0.5}
%\definecolor{codepurple}{rgb}{0.58,0,0.82}
%\definecolor{backcolour}{rgb}{0.95,0.95,0.92}

% Parametros para setear como se muestra el codigo

\lstdefinestyle{mystyle}{                       
    backgroundcolor=\color{backcolour},
    commentstyle=\color{codegreen},
    keywordstyle=\color{magenta},
    numberstyle=\tiny\color{codegray},
    stringstyle=\color{codepurple},
    basicstyle=\ttfamily\scriptsize,
    breakatwhitespace=false,         
    breaklines=true,                 
    captionpos=b,                    
    keepspaces=true,                 
    numbers=left,                    
    numbersep=5pt,                  
    showspaces=false,                
    showstringspaces=false,
    showtabs=false,                  
    tabsize=4
}

\lstset{style=mystyle}

% Macro para divisores horizontales

\newcommand{\divider}                           
{
\begin{center}
    \hrulefill
\end{center}
\vspace{0.25cm}
\normalsize
}

% Macro para comillas

\newcommand{\quotes}[1]{``#1''}                 % Comando para double quotes

% Números de figura y ecuación por seccion (1.1 , 2.1, etc.)

\numberwithin{figure}{section}                  % Numeros de figura por seccion
\numberwithin{equation}{section}                % Numeros de ecuacion por seccion

%---------------------------------------------------------------------------------------------------------%
%------------------------------------------ Inicio de Documento ------------------------------------------%
%---------------------------------------------------------------------------------------------------------%

\begin{document}

    \nocite{*}                                  % Se usa para que aparezcan todas la referencias del .bib sin tener que 
                                                % citarlas en el texto
    \newpagecolor{AzulFI}\afterpage{\restorepagecolor\blankpage}    % Página en blanco entre portada y agradecimientos

        \begin{titlepage}
        \begin{center}
            \vspace*{0.5cm}
            \Huge
            \textbf{Diseño y desarrollo de una plataforma experimental de evaluación de sistemas híbridos basados en pilas de combustible}    % Titulo
            \\
            \vspace{0.5cm}
            \huge
            Proyecto Final                                       % Subtitulo
            \\
            \vspace{2cm}
            \Large
            \textbf{Autor:}
            \\
            \large
            \vspace{0.2cm}
            Tomás Tavella - 68371/4
            \\
            \vspace{1cm}
            \Large
            \textbf{Director:}
            \\
            \vspace{0.2cm}
            \large
            Ing. Jorge Anderson Azzano
            \\
            \vspace{0.3cm}
            \Large
            \textbf{Co-director:}
            \\
            \vspace{0.2cm}
            \large
            Dr. Ing. Paul F. Puleston
            \\
            \vfill
            \begin{figure}[H]
                \centering
                \begin{subfigure}
                    \centering
                    \includegraphics[width=0.25\textwidth]{Imagenes/UNLP.pdf}
                \end{subfigure}
                \begin{subfigure}
                    \centering
                    \includegraphics[width=0.32\textwidth]{Imagenes/FI.jpg}
                \end{subfigure}
            \end{figure}
            \vspace{1cm}
            \textit{
            Facultad de Ingeniería
            \\
            Universidad Nacional de La Plata}
            \vspace{1cm}
        \end{center}
    \end{titlepage}
                          
    \newpage 
    \thispagestyle{empty}                       % Para que no se muestre el número de página al final (igual contribuye a la cuenta total)
    \afterpage{\blankpage}

    \Huge
\textbf{Agradecimientos}\\

\normalsize
Lorem ipsum dolor sit amet, consectetur adipiscing elit, sed do eiusmod tempor incididunt ut labore et dolore magna aliqua. Varius quam quisque id diam vel quam. Morbi tristique senectus et netus et malesuada fames ac. Fermentum leo vel orci porta non. Ullamcorper morbi tincidunt ornare massa eget egestas purus viverra accumsan. Non quam lacus suspendisse faucibus. Facilisis volutpat est velit egestas. In mollis nunc sed id semper risus. Lobortis mattis aliquam faucibus purus in massa tempor nec feugiat. Pellentesque elit eget gravida cum sociis natoque penatibus et. Aenean et tortor at risus viverra adipiscing at. Nunc sed blandit libero volutpat. Pretium fusce id velit ut. Sed faucibus turpis in eu.


    \newpage
    \thispagestyle{empty}
    \afterpage{\blankpage}
    \addtocounter{page}{+2}

    \huge
\scshape
\textbf{Resumen}\\

\normalfont\normalsize
Este trabajo consiste del estudio, diseño, implementación y validación de una plataforma experimental para la evaluación de sistemas híbridos de generación energía (SHGE) a partir de pilas o celdas de combustible de tipo PEMFC (\textit{Proton Exchange Membrane Fuel Cell}). Esta plataforma consiste en un sistema de conversión electrónico de tipo CC-CC conmutado y aislado, de topología puente completo; monitoreado mediante la medición de sus estados, y controlado por una excitación de tipo PWM (\textit{Pulse-Width Modulation}) provista por un DSC (\textit{Digital Signal Controller}) de alta performance. Este conversor es requerido para poder adaptar la tensión variable que entrega una celda de combustible a una tensión de salida fija para conectar a un bus común de corriente continua.\\

En el desarrollo de este informe se detallan las tareas realizadas para cumplir este objetivo: el estudio y comprensión de las topologías de conversión CC-CC; la simulación de la topología elegida mediante herramientas de simulación circuitales; el diseño de circuitos auxiliares de excitación, sensado y protección; la implementación del sistema en una placa de circuito impreso mediante software EDA (\textit{Electronic Design Automation}); la programación de los algoritmos de control del sistema; y, finalmente la validación experimental de la plataforma.\\

\vspace{1cm}
\huge
\scshape
\textbf{Abstract}\\

\normalsize\normalfont
This work entails the study, design, implementation and validation of an experimental platform for the evaluation of hybrid energy generation systems based on Proton Exchange Membrane Fuel Cells (PEMFC). This platform incorporates a full-bridge isolated switched-mode DC-DC electronic converter, monitored via the measurement of its state variables, and controlled by a pulse-width modulated (PWM) signal, generated using a high-performance Digital Signal Controller (DSC). This converter provides the adaptation from the variable output voltage of the PEMFC to the fixed voltage of the common DC bus at the system output.\\

This report details the process through which the goals were achieved: study and understanding of the different DC-DC converter topologies, simulation of the selected converter topology using circuit simulation tools, design process of auxiliary circuits, including driver, sensing and protection circuits,  implementation of the system PCB (printed circuit board) through the use of electronic design automation (EDA) software, programming of system control algorithms, and experimental validation of the working platform.\\ 

    \newpage
    \afterpage{\blankpage} 
    \tableofcontents
    \newpage

    \section{Introducción}

\vspace{0.5cm}

\Large\scshape
Sistema completo en el que se engloba la plataforma de evaluación en estudio. Contexto global como justificación
\normalfont

\divider

Lorem ipsum dolor sit amet, consectetur adipiscing elit, sed do eiusmod tempor incididunt ut labore et dolore magna aliqua. Aliquet enim tortor at auctor urna. Ac orci phasellus egestas tellus rutrum tellus pellentesque eu. Aliquam eleifend mi in nulla. Sit amet cursus sit amet dictum sit amet justo. Tellus orci ac auctor augue mauris augue neque gravida in. Tincidunt dui ut ornare lectus sit amet est. Nulla facilisi morbi tempus iaculis urna id. Quis vel eros donec ac odio tempor orci dapibus. Sed cras ornare arcu dui vivamus. Augue neque gravida in fermentum et. At urna condimentum mattis pellentesque id nibh tortor id. Malesuada fames ac turpis egestas integer eget. Nec feugiat in fermentum posuere urna nec. Pellentesque pulvinar pellentesque habitant morbi. Nunc sed id semper risus in hendrerit gravida.\\

\subsection{Subsección 1}

Parturient montes nascetur ridiculus mus. Pulvinar etiam non quam lacus suspendisse faucibus. Fusce id velit ut tortor pretium viverra suspendisse potenti nullam. Porta non pulvinar neque laoreet suspendisse. Pellentesque id nibh tortor id aliquet lectus. Semper viverra nam libero justo. Vitae tortor condimentum lacinia quis vel eros donec. Ullamcorper velit sed ullamcorper morbi tincidunt. Pellentesque habitant morbi tristique senectus et netus. Non curabitur gravida arcu ac tortor dignissim convallis aenean. Fringilla urna porttitor rhoncus dolor purus non enim praesent. Eget aliquet nibh praesent tristique magna sit amet purus gravida. Orci porta non pulvinar neque. Id porta nibh venenatis cras sed felis. Id neque aliquam vestibulum morbi blandit cursus risus at.\\

\subsection{Subsección 2}

In iaculis nunc sed augue lacus. Odio ut enim blandit volutpat maecenas volutpat. Cras sed felis eget velit aliquet. Risus in hendrerit gravida rutrum quisque non. Risus in hendrerit gravida rutrum quisque non tellus orci. Nec ullamcorper sit amet risus nullam eget felis. Gravida arcu ac tortor dignissim convallis aenean et tortor at. Vehicula ipsum a arcu cursus vitae congue mauris rhoncus. Montes nascetur ridiculus mus mauris vitae ultricies leo integer malesuada. Bibendum arcu vitae elementum curabitur. Vel risus commodo viverra maecenas accumsan lacus vel. Aliquet nec ullamcorper sit amet risus nullam eget felis. Amet volutpat consequat mauris nunc congue nisi vitae. Ultrices tincidunt arcu non sodales neque sodales. Sed odio morbi quis commodo. Cursus risus at ultrices mi tempus imperdiet. Scelerisque eu ultrices vitae auctor eu augue.\\
    
    \newpage

    \section{Plataforma de Evaluación}

\vspace{0.5cm}

\Large\scshape
\begin{center}
    Análisis de la plataforma de evaluación de celdas de combustible
\end{center}
\normalfont

\divider

En este capítulo, se realiza un detallado análisis de la Plataforma Experimental de Evaluación de Módulos de Celdas de Combustible de la figura \ref{diag_plataforma}, la cuál consiste en cuatro subsistemas o bloques distintos: 

\begin{itemize}
    \item Emulador de Celdas de Combustible
    \item Conversor CC-CC Conmutado
    \item Sistema de Control
    \item Carga Electrónica Variable
\end{itemize}

\begin{figure}[h]
    \centering
    \includegraphics[scale=0.4]{Imagenes/Plataforma Experimental.pdf}
    \caption{Diagrama de la plataforma experimental de evaluación, con sus cuatro bloques principales.}
    \label{diag_plataforma}
\end{figure}

Esta plataforma, con sus distintos bloques, se encarga de evaluar la \textit{performance} de celdas de combustible conectadas a un sistema híbrido de generación. Con este fin, un emulador de celdas de combustible toma el puesto de celdas de combustible reales, y una carga electrónica variable se utiliza para simular cualquier tipo de condiciones de carga que se deseen en el bus de CC. Para poder conectar el emulador a la carga, se debe implementar un subsistema (Conversor CC-CC Conmutado) que adapte los niveles de tensión de salida del emulador de celdas a la tensión fija de salida en la carga, adicionando un módulo de control que monitorea los estados del conversor, y los controla mediante los disparos de las llaves del puente completo.\\

El principal objetivo de este proyecto es el diseño e implementación de la etapa de adaptación de tensión (es decir, el conversor con su sistema de control), pero se hace un estudio detallado de todas los componentes de la plataforma, de manera de obtener un entendimiento más completo de todo el sistema. Por esta razón, a continuación se hace un análisis en profundidad de cada una de las partes individuales, comenzando por el emulador de celdas de combustible.\\

\subsection{Emulador de Celdas de Combustible}

A pesar de que las celdas de combustible son una tecnología de hace más de un siglo y medio (desarrollada por primera vez por el físico galés Sir William Grove en 1842), hoy en día despiertan un particular interés en el campo de la generación renovable por su alta eficiencia, su dependencia en recursos obtenibles fácilmente de maneras ambientalmente amigables, y por la generación de agua como único deshecho.\\

Por estas razones se eligió trabajar con esta tecnología, particularmente con el tipo de celda más común hoy en día, las Celdas de Combustible de Membrana de Intercambio Protónico o PEM-FC (del inglés \textit{Proton Exchange Membrane Fuel Cell}), cuyo funcionamiento se profundiza más adelante.\\

\subsubsection{Principio de Funcionamiento de las Celdas de Combustible}

Esencialmente, una celda de combustible es una celda galvánica o celda voltáica en la cual la energía libre de una reacción química redox (entre un combustible y un agente oxidante) se convierte a energía eléctrica mediante una corriente y una diferencia de potencial$^{[FC-FundamentalsAndApplications]}$. En nuestro caso particular, el combustible es el hidrógeno molecular ($H_2$), el agente oxidante es el oxígeno ($O_2$) abundante en la atmósfera, y los productos son la energía eléctrica y el agua ($H_2O$) según indica la siguiente ecuación química balanceada.

\begin{equation}\label{redox_celda}
    2H_2\ +\ O_2\ \longrightarrow\ 2H_2O
\end{equation}

La estructura interna de una celda de combustible, visible en la figura \ref{fuel_cell}, consiste de un ánodo (electrodo negativo) al cual ingresan las moléculas de hidrógeno, un cátodo (electrodo positivo) en el que ingresa el oxígeno y se despide el agua, y un electrolito como como interfaz entre ánodo y cátodo. La carga es conectada entre el ánodo y el cátodo.

\begin{figure}[h]
    \centering
    \includegraphics[scale=0.35]{Imagenes/Fuel Cell.png}
    \caption{Esquema de una celda de combustible, con todos sus componentes indicados (Placeholder).}
    \label{fuel_cell}
\end{figure}

La reacción redox de la ecuación \ref{redox_celda}, dentro de una celda de combustible como la del esquema, en realidad se separa en dos reacciones parciales distintas:

\begin{equation}\label{redox_anodo}
    2H_2\ \longrightarrow\ 4H^{+}\ +\ 4e^-
\end{equation}

\begin{equation}\label{redox_catodo}
    4H^{+}\ +\ 4e^-\ +\ O_2\longrightarrow\ 2H_2O
\end{equation}

De esta manera, alimentado simultáneamente el terminal negativo con combustible (hidrógeno) y el terminal positivo con oxidante (oxígeno) se producen las dos reacciones en las superficies de contacto del electrolito:

\begin{itemize}
    \item \textbf{En el ánodo} las moléculas de $H_2$ pierden sus electrones, bifurcándose los iones positivos de hidrógeno ($H^{+}$) por el electrolito y los electrones libres a través de la carga (ecuación \ref{redox_anodo}). Es una reacción exotérmica (libera calor) que resulta en el calentamiento de la celda.
    \item \textbf{En el cátodo} los iones $H^{+}$ del electrolito, los electrones libres, y las moléculas de oxígeno reaccionan para formar como producto el agua (ecuación \ref{redox_catodo}).
\end{itemize}

Mediante este proceso electroquímico se generan dos corrientes distintas: una corriente interna de iones $H^{+}$ (cargas positivas) en el electrolito, desde el ánodo hacia el cátodo; y una corriente externa de electrones $e^-$ (cargas negativas) circulando por la carga, en el mismo sentido que la corriente de iones. Esta última corriente de electrones es la que nos resulta útil para poder alimentar algún tipo de carga.\\

\subsubsection{Aspectos Constructivos de una Celda}



\subsubsection{De Celda a Pila de Combustible}

Sin embargo, una celda de combustible individual como en la figura \ref{fuel_cell} no es capaz de entregar una diferencia de potencial lo suficientemente alta para la gran mayoría de las aplicaciones, con una tensión de celda común situada entre 0.7 V y 1.3 V, dependiendo de varios aspectos constructivos específicos de la celda.\\

    \newpage

    %\printbibliography 
    
\end{document}