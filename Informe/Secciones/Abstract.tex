\huge
\scshape
\textbf{Resumen}\\

\normalfont\normalsize
Este trabajo consiste del estudio, diseño, implementación y validación de una plataforma experimental para la evaluación de sistemas híbridos de generación energía (SHGE) a partir de pilas o celdas de combustible de tipo PEMFC (\textit{Proton Exchange Membrane Fuel Cell}). Esta plataforma consiste en un sistema de conversión electrónico de tipo CC-CC conmutado y aislado, de topología puente completo; monitoreado mediante la medición de sus estados, y controlado por una excitación de tipo PWM (\textit{Pulse-Width Modulation}) provista por un DSC (\textit{Digital Signal Controller}) de alta performance. Este conversor es requerido para poder adaptar la tensión variable que entrega una celda de combustible a una tensión de salida fija para conectar a un bus común de corriente continua.\\

En el desarrollo de este informe se detallan las tareas realizadas para cumplir este objetivo: el estudio y comprensión de las topologías de conversión CC-CC; la simulación de la topología elegida mediante herramientas de simulación circuitales; el diseño de circuitos auxiliares de excitación, sensado y protección; la implementación del sistema en una placa de circuito impreso mediante software EDA (\textit{Electronic Design Automation}); la programación de los algoritmos de control del sistema; y, finalmente la validación experimental de la plataforma.\\

\vspace{1cm}
\huge
\scshape
\textbf{Abstract}\\

\normalsize\normalfont
This work entails the study, design, implementation and validation of an experimental platform for the evaluation of hybrid energy generation systems based on Proton Exchange Membrane Fuel Cells (PEMFC). This platform incorporates a full-bridge isolated switched-mode DC-DC electronic converter, monitored via the measurement of its state variables, and controlled by a pulse-width modulated (PWM) signal, generated using a high-performance Digital Signal Controller (DSC). This converter provides the adaptation from the variable output voltage of the PEMFC to the fixed voltage of the common DC bus at the system output.\\

This report details the process through which the goals were achieved: study and understanding of the different DC-DC converter topologies, simulation of the selected converter topology using circuit simulation tools, design process of auxiliary circuits, including driver, sensing and protection circuits,  implementation of the system PCB (printed circuit board) through the use of electronic design automation (EDA) software, programming of system control algorithms, and experimental validation of the working platform.\\ 