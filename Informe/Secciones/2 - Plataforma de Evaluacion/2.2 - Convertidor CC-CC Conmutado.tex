\subsection{Convertidor CC-CC Conmutado}

Un convertidor CC-CC es un dispositivo electrónico que tiene como objetivo convertir una tensión continua $V_{in}$ a la entrada, a una tensión continua $V_{out}$ de distinta magnitud a la salida, transfiriendo la mayor cantidad de energía posible de la entrada hacia la salida. Dependiendo del tipo de convertidor, esta tensión de salida puede ser menor, mayor o tanto menor como mayor a la tensión de entrada. Estos dispositivos son necesarios en casos, como el nuestro, en el que los niveles de tensión de entrada y salida no son compatibles y deben ser adaptados.\\

La forma más básica que se podría concebir para un dispositivo que cumpla esta función es la de un simple divisor resistivo, en el cual la tensión $V_{out}$ depende de las resistencias $R_1$ y $R_2$ y la tensión de entrada $V_{in}$.

\begin{equation*}
    V_{out} = V_{in}\cdot\frac{R_2}{R_1+R_2}
\end{equation*}

Entonces, variando la relación entre $R_1$ y $R_2$, se puede variar la tensión $V_{out}$ entre tensión nula y $V_{in}$. Sin embargo, se necesita solo un análisis superficial de esta topología para ver que no es viable para ningún tipo de aplicación, más que nada por su pobre eficiencia energética (para obtener una tensión igual a la mitad de la entrada, se pierde la mitad de la potencia en disipación resistiva).\\

Los convertidores CC-CC se suelen separar en dos principales categorías: los {\Medium reguladores lineales}, que son un caso complejizado del divisor resistivo donde se utiliza un transistor como resistencia varibale (además de un diodo para regular la tensión de salida); y los {\Medium convertidores conmutados}, en los cuales uno o más transistores, actuando como llaves, son conmutados a alta frecuencia y con dispositivos que almacenan energía (como inductores y capacitores) se obtiene una tensión continua a la salida.\\

{\Thin Thin} {\ExtraLight ExtraLight} {\Light Light} Regular {\Medium Medium}  {\SemiBold SemiBold} {\Bold Bold} {\ExtraBold ExtraBold} {\Black Black}