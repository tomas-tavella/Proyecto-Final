\subsection{Resumen}

En este capítulo se detalló el proceso mediante el cual se pasó de un diseño circuital teórico para la plataforma de evaluación de sistemas híbridos obtenido en el capítulo anterior, a una PCB real de más de 180 componentes discretos que implementa el sistema.\\

Primero se trató el diseño de la PCB en el \textit{PCB Editor} de KiCad partiendo del circuito diseñado, eligiendo footprints para cada componente, ordenándolos de manera lógica en el área física de la placa, trazando y dimensionado las pistas de cobre que interconectan cada componente. Con el diseño finalizado, se enviaron los archivos de fabricación a una fábrica local de circuitos impresos.\\

Una vez obtenidas las placas producidas por el fabricante de acuerdo a las especificaciones del diseño, se procedió a soldar cada uno de los componentes, así obteniendo una placa finalizada y funcional. Además, se detallaron los inconvenientes importantes que surgieron y se debieron resolver a lo largo del proceso.\\

Con los resultados de este capítulo, la plataforma se encuentra lista para los ensayos que verificarán su funcionamiento correcto en el siguiente capítulo.\\