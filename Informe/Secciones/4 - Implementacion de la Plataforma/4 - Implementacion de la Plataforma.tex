\section{Implementación de la Plataforma} \label{implementacion}
\AddToShipoutPictureBG*{\includegraphics[width=\paperwidth,height=\paperheight]{Imagenes/Fondo Capitulo 4.pdf}}
\thispagestyle{plain}

\vspace{0.5cm}

\Large\scshape
\begin{center}
    \textrm{Implementación de los circuitos en la placa de circuito impreso}
\end{center}
\normalfont
%\normalsize

\divider

En el capítulo anterior, se diseñaron todos los circuitos que forman la plataforma, y se dimensionaron eléctricamente y físicamente todos sus componentes. Sin embargo, estos esquemas no cuentan con información del posicionamiento físico o \textit{layout} de los componentes, simplemente representan las conexiones eléctricas que vinculan los distintos componentes. Para poder llegar a una implementación física en una placa de circuito impreso, se deben posicionar los elementos teniendo en cuenta su tamaño, y conectarlos mediante pistas de cobre en un espacio lo más compacto posible.\\

\begin{figure}[h]
    \centering
    \includegraphics[scale=0.3]{Imagenes/PCB 3D Raytracing.png}
    \caption{Modelo tridimensional de la implementación en PCB de la plataforma con todos sus componentes, vista desde la parte superior.}
    \label{pcb_3d}
\end{figure}

Además de los componentes que se mencionaron o fueron mostrados en el proceso de diseño, existen otros componentes como conectores, borneras, pulsadores, LEDs y algunas resistencias y capacitores que se omitieron, ya que conforman únicamente detalles que incumben solamente a la implementación física. Teniendo en cuenta esto, el total de componentes discretos de la PCB llega a los 185. Se puede observar un modelo tridimensional de la placa final con todos sus componentes, generado por KiCad.\\

\subsection{Diseño de PCB}

\subsubsection{Dimensiones y Composición}

\lipsum[2]\\

\subsubsection{Asignación de Footprints}

\lipsum[3]\\

\subsubsection{Posicionamiento de Componentes}

\lipsum[4]\\

\subsubsection{Trazado y Ancho de Pistas}

\lipsum[5]\\

\subsubsection{Fabricación y Resultado}

\lipsum[6]\\

\newpage

\subsection{Soldado de Componentes}

Una vez que las placas fabricadas arribaron al laboratorio como en la figura \ref{placa_fisica} y todos los componentes de la plataforma se encontraban disponibles en el laboratorio, se comenzó el proceso de soldar el total de 180+ componentes discretos a la placa, que se completó en un marco de tiempo de aproximadamente tres semanas.\\

\begin{figure}[h]
    \centering
    \includegraphics[scale=0.09]{Imagenes/Soldado.jpg}
    \caption{Zona de trabajo dónde se realizó el soldado de los componentes. A la derecha se puede observar la estación de soldado de temperatura regulable.}
    \label{soldado}
\end{figure}

Con asistencia del director del proyecto se aprendieron las nociones básicas de soldado de componentes SMD, y se completó todo el proceso sin mayores inconvenientes. Se comenzó soldando todos los componentes necesarios para evaluar el correcto funcionamiento del circuito driver y la excitación PWM, que incluyen los dos drivers medio puente de 2ED21834-S06J y todos su circuito auxiliar, los optoacopladores ACPL-P480 para aislar el DSC, el conector \textit{barrel} para la alimentación externa y el regulador lineal LM7805 para alimentar los optoacopladores.\\

Luego se continuó soldando el resto de los componentes, dejando los de mayores dimensiones físicas, en particular los capacitores de entrada y salida del convertidor, para el final del proceso, de manera que no estorben para la colocación de otros componentes.\\

El único componente que presento alguna dificultad al soldar fue el sensor LM5056A, cuyo empaquetado HTSSOP-28 cuenta con un pad encontrado en la parte inferior. En este caso, este pad es simplemente una conexión a tierra y cumple el único propósito de mejorar la capacidad de disipación térmica del dispositivo. Como no se cuenta con el equipo apropiado para soldar esta conexión, se utilizó una pasta térmica no conductora para al menos conseguir algín nivel de transferencia de calor del sensor al plano de tierra.\\

\newpage

\subsection{Contratiempos}

Sin embargo, este proceso de diseño y fabricación de PCB no fue sin sus contratiempos y errores, que hubieron de ser resueltos previamente a poder realizar pruebas y ensayos con la placa. Estos inconvenientes fueron dados por falta de experiencia en el ámbito de diseño de PCB y utilización de software EDA, en especial para el diseño de una placa tan compleja.\\

Muchos de estos errores fueron detectados previamente al proceso de fabricación y pudieron ser corregidos en el esquemático de PCB dentro de KiCad. Por ejemplo, en el diseño inicial del circuito para las ondas PWM de los transistores, se habían utilizado los puertos de ePWM incorrectos que hubiesen dificultado la implementación del control de ciclo de trabajo mediante phase-shift, pero en una revisión de los archivos de diseño previo a la fabricación se encontró y rectificó este error.\\

\subsubsection{Footprint de MOSFET}

SEl error de diseño más importante en la placa se descubrió una vez que ya habían arribado las placas impresas al laboratorio. Durante las fases tempranas del diseño de la placa se colocaron las cuatro footprints TO-247 correspondientes a los MOSFET IRFP150, pero por desconocimiento, se elaboró su footprint en base a la modificación de otra, en vez de obtenerla de las bibliotecas online ya mencionadas.\\

La footprint para estos transistores que se encuentra sobre la placa tiene sus terminales colocados en el orden \textit{Drain-Gate-Source}. Sin embargo, el IRFP150 y todos los MOSFET de potencia de paquete TO-247 tienen el terminal Gate en la primera posición, con el orden \textit{Gate-Drain-Source}.\\

\paragraph{Placa Adaptadora}

Se evaluaron varias soluciones para este problema, incluso buscando modelos alternativos de MOSFET de potencia que tuvieran los terminales en el orden necesario. La solución por la que se optó fue diseñar una pequeña placa adaptadora simple faz que se conecte en la posición original de los transistores, y con las pistas de cobre invertir la posición de lo terminales de drain y gate.\\

\begin{figure}[h]
    \centering
    \includegraphics[scale=1.2]{Imagenes/Placa Adaptadora.pdf}
    \caption{Diagrama de la capa de cobre de la placa adaptadora diseñada para mitigar el error de diseño.}
    \label{placa_adaptadora}
\end{figure}

En su diseño se tuvo que buscar hacer la placa lo más pequeña posible, ya que los transistores tienen borneras y otros componentes alrededor que limitan cuan grande puede ser. Además, la nueva fooprint de los MOSFET debía quedar lo más cercana posible al borde de la placa, para facilitar el montaje del disipador de los transistores.\\

Una vez que se finalizó el diseño, que no llevó más de un par de días, se utilizaron las instalaciones disponibles en el ATEI (Área Técnica de Electrónica e Instrumental), y con una placa de cobre simple faz, se fabricó la placa adaptadora que luego se soldó por sobre la placa original.\\

\newpage

\subsection{Resumen}

En este capítulo se detalló el proceso mediante el cual se pasó de un diseño circuital teórico para la plataforma de evaluación de sistemas híbridos obtenido en el capítulo anterior, a una PCB real de más de 180 componentes discretos que implementa el sistema.\\

Primero se trató el diseño de la PCB en el \textit{PCB Editor} de KiCad partiendo del circuito diseñado, eligiendo footprints para cada componente, ordenándolos de manera lógica en el área física de la placa, trazando y dimensionado las pistas de cobre que interconectan cada componente. Con el diseño finalizado, se enviaron los archivos de fabricación a una fábrica local de circuitos impresos.\\

Una vez obtenidas las placas producidas por el fabricante de acuerdo a las especificaciones del diseño, se procedió a soldar cada uno de los componentes, así obteniendo una placa finalizada y funcional. Además, se detallaron los inconvenientes importantes que surgieron y se debieron resolver a lo largo del proceso.\\

Con los resultados de este capítulo, la plataforma se encuentra lista para los ensayos que verificarán su funcionamiento correcto en el siguiente capítulo.\\

\afterpage{\blankpage}\newpage