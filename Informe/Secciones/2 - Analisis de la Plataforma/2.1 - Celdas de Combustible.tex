\subsection{Celdas de Combustible}

A pesar de que las celdas de combustible son una tecnología de hace más de un siglo y medio (desarrollada por primera vez por el físico galés Sir William Grove en 1842), hoy en día despiertan un particular interés en el campo de la generación renovable por su alta eficiencia, su dependencia en recursos obtenibles fácilmente de maneras ambientalmente amigables, y la generación de agua como único deshecho.\\

Por estas razones se eligió trabajar con esta tecnología, particularmente con el tipo de celda más común hoy en día, las Celdas de Combustible de Membrana de Intercambio Protónico o PEMFC (del inglés \textit{Proton Exchange Membrane Fuel Cell}), cuyo funcionamiento se profundiza más adelante.\\

\subsubsection{Principio de Funcionamiento}

Esencialmente, una celda de combustible es una celda galvánica o celda voltáica en la cual la energía libre de una reacción química redox (entre un combustible y un agente oxidante o \textit{comburente}) se convierte a energía eléctrica mediante una corriente y una diferencia de potencial.\textsuperscript{\cite{FC-FundAndAppl}} En nuestro caso particular, el combustible es el hidrógeno molecular ($H_2$), el agente oxidante es el oxígeno ($O_2$) abundante en la atmósfera, y los productos son la energía eléctrica y el agua ($H_2O$) según indica la siguiente ecuación química balanceada.

\begin{equation}\label{redox_celda}
    H_2\ +\ \frac{1}{2}O_2\ \longrightarrow\ H_2O
\end{equation}

La estructura interna de una celda de combustible, visible en la figura \ref{fuel_cell}, consiste de un ánodo (electrodo negativo) al cual ingresan las moléculas de hidrógeno, un cátodo (electrodo positivo) en el que ingresa el oxígeno y se despide el agua, y un electrolito como como interfaz entre ánodo y cátodo. La carga es conectada entre el ánodo y el cátodo.\\

\begin{figure}[h]
    \centering
    \includegraphics[scale=0.25]{Imagenes/Fuel Cell.png}
    \caption{Esquema ilustrativo de una celda de combustible, con todos sus componentes indicados.}
    \label{fuel_cell}
\end{figure}

La reacción redox de la ecuación \ref{redox_celda}, dentro de una celda de combustible como la del esquema, en realidad se separa en dos reacciones parciales distintas.

\begin{equation}\label{redox_anodo}
    H_2\ \longrightarrow\ 2H^{+}\ +\ 2e^-
\end{equation}

\begin{equation}\label{redox_catodo}
    2H^{+}\ +\ 2e^-\ +\ \frac{1}{2}O_2\longrightarrow\ H_2O
\end{equation}

De esta manera, alimentado simultáneamente el terminal negativo con combustible (hidrógeno) y el terminal positivo con oxidante (oxígeno) se producen las dos reacciones en las superficies de contacto del electrolito:

\begin{itemize}
    \item {\SemiBold En el ánodo ocurre la oxidación:} las moléculas de $H_2$ pierden sus electrones, bifurcándose los iones positivos de hidrógeno ($H^{+}$) por el electrolito y los electrones libres a través de la carga (ecuación \ref{redox_anodo}). Es una reacción exotérmica (libera calor) que resulta en el calentamiento de la celda.
    \item {\SemiBold En el cátodo ocurre la reducción:} los iones $H^{+}$ del electrolito, los electrones libres, y las moléculas de oxígeno reaccionan para formar como producto el agua (ecuación \ref{redox_catodo}).
\end{itemize}

Mediante este proceso electroquímico se generan dos corrientes distintas: una corriente interna de iones $H^{+}$ (cargas positivas) en el electrolito, desde el ánodo hacia el cátodo; y una corriente externa de electrones $e^-$ (cargas negativas) circulando por la carga, en el mismo sentido que la corriente de iones. Esta última corriente de electrones es la que nos resulta útil para poder alimentar algún tipo de carga.\\

\subsubsection{De Celda a Pila de Combustible}

Sin embargo, una celda de combustible individual como en la figura \ref{fuel_cell} no es capaz de entregar una diferencia de potencial lo suficientemente alta para la gran mayoría de las aplicaciones, con una tensión de celda común situada entre \SI{0.7}{\volt} y \SI{1.3}{\volt}, dependiendo de varios aspectos constructivos específicos de la celda.\\

Entonces, para obtener un dispositivo con una tensión de salida de niveles utilizables, esta tecnología generalmente se comercializa en forma de pilas o \textit{stacks} de celdas individuales conectadas en serie como se ve en la figura \ref{fuel_cell_stack}, generalmente de entre diez y cien celdas, cuya tensión es la suma de la tensión de cada celda que la compone.\\

Esto se logra, como dice su nombre, apilando todas las celdas de combustible para formar el \textit{stack}, utilizando placas de interconexión para conectar electrodos de polaridad opuesta de dos celdas aledañas (es decir, se conecta el ánodo de una celda con el cátodo de la siguiente); además de cumplir la función de aislar el combustible de una celda del agente oxidante de la celda contigua. Este es el tipo de conexionado de celdas más común, llamado \textit{Planar-Bipolar Stacking} o Apilado Planar-Bipolar.\textsuperscript{\cite{FCHandbook}}\\

\begin{figure}[h]
    \centering
    \includegraphics[scale=0.5]{Imagenes/Fuel Cell Stack.png}
    \caption{Diagrama interno de un stack de celdas conectadas por apilado planar-bipolar.}
    \label{fuel_cell_stack}
\end{figure}

\subsubsection{Aspectos Constructivos}

Habiendo repasado el principio básico de funcionamiento de las celdas de combustible, ahora se realizará una breve descripción de los aspectos constructivos de las mismas. La utilización de distintos materiales y composiciones de las partes que las componen derivan en distintos tipos de celdas, que, a pesar de funcionar bajo el mismo principio básico, poseen cada una sus ventajas y desventajas que las hacen más o menos apropiadas para distintas aplicaciones.\\

Como las reacciones químicas ocurren en superficies microscópicas dónde alguno de los electrodos está en contacto con el electrolito, generalmente los electrodos se fabrican de materiales porosos que aumentan la posible superficie de contacto entre ambas fases, acelerando las reacciones necesarias para producir energía. Sin embargo, en muchos casos, a temperaturas bajas los materiales de los electrodos no son capaces de producir la suficiente actividad electroquímica, por lo que suelen agregarse pequeñas cantidades de catalizador en las zonas de contacto para acelerar la reacción.\\

En tanto al electrolito, estos suelen estar hechos de materiales en estado líquido o sólido, dependiendo del tipo de celda, pero siempre deben tener una alta conductividad de iones positivos, de manera que los iones $H^{+}$ circulen solo por el elctrolito y no por el circuito externo. Adicionalmente, este material debe actuar de barrera física para evitar que se mezclen los flujos de combustible y comburente.\\

En tanto a la geometría de las celdas, se ha expermientado con una gran variedad de formas para los electrodos y electrolitos pero, hoy en día las pilas que se producen son mayormente planas, y en algunos casos tubulares.\\

\subsubsection{Tipos de Celdas}\label{tipos_celdas}

Hay muchas formas de clasificar las distintas tecnologías de celdas, pero en nuestro caso nos vamos a enfocar en la distinción más común, que es la clasificación según el material usado como electrolito. Hoy en día, hay seis tipos distintos de celdas segun electrolito, descritas a continuación, con una mayor profundización mayor en las del tipo PEMFC que se mencionaron anteriormente, ya que son este tipo de pilas las que nos interesa en nuestra aplicación particular.\\

\paragraph{Celda de Combustible Alcalina (AFC)}

Las AFC fueron las primeras celdas de combustible en ser desarrolladas, alrededor de 1960, e incluso hoy en día son las celdas de combustible con la mayor eficiencia eléctrica. Sin embargo, resultan poco viables, principalmente porque requieren gases muy puros para funcionar correctamente. Este requerimiento se da por el material electrolítico utilizado, el Hidróxido de Potasio (KOH) (en concentraciones de 85 \% para celdas de alta temperatura (\SI{250}{\celsius}), y entre 35 \% y 50 \% para celdas de baja temperatura (<\SI{120}{\celsius})), que reacciona facilmente con el dióxido de carbono que abunda en el aire, transformándose en $K_2CO_3$, destruyendo el electrolito y la celda en el proceso.\textsuperscript{\cite{FC-FundAndAppl}\cite{FCHandbook}}\\

\paragraph{Celda de Combustible de Membrana de Intercambio Protónico (PEMFC)}

Las PEMFC, también llamadas Celdas de Combustible de Electrolito Polimérico Sólido (SPEFC) son las celdas de combustible más utilizadas al día de hoy, habiendo conseguido usos en vehículos de combustible alternativo, lo que resultó en una gran inversión para su desarrollo. Estas celdas operan en rangos bajos de temperatura (entre \SI{65}{\celsius} y \SI{105}{\celsius}) y tienen un electrolito de estado sólido.\\
    
Este electrolito es una membrana de intercambio protónico: un membrana semipermeable que permite la conducción de protones y al mismo tiempo funcionando de aislación eléctrica entre los electrodos, y como barrera física para separar el combustible del comburente. Esta membrana solía fabricarse de sulfonato de poliéstireno, pero hoy en día se usan materiales basados en Politetrafluoretileno (PTFE) como el Nafion de DuPont o el Dow de Dow Chemical, que son más estables y poseen mayor conductividad de protones.\\

Su baja temperatura de operación, uso de materiales no exóticos, capacidad de altas densidades de corriente, resistencia a la corrosión dada por el electrolito sólido y bajo tiempo de arranque han hecho a las PEMFC la opción más popular al elegir un tipo de celda de combustible para utilizar. Sin embargo tiene sus desventajas, como el angosto rango de temperatura en el que requiere operar.\textsuperscript{\cite{FC-FundAndAppl}\cite{FCHandbook}}\\

Como esta es la tecnología de celda que nos interesa, se va a dedicar una sección para continuar más detalladamente la descripción de este tipo de celdas.\\

\paragraph{Celda de Combustible de Metanol Directo (DMFC)}

Las DMFC son un tipo especial de celdas de baja temperatura basadas en tecnología de las PEMFC, operando a temperaturas ligeramente mayores a estas. A diferencia de otras tecnologías, estas celdas utilizan metanol como combustible directamente, ahorrándose el paso de reformarlo a hidrógeno. El metanol es un combustible atractivo, ya que se puede producir a partir de gas natural o biomasa renovable y tiene una elevada energía específica.\textsuperscript{\cite{FC-FundAndAppl}\cite{FCHandbook}}\\

\paragraph{Celda de Combustible de Carbonato Fundido (MCFC)}

Las MCFC, desarrolladas a mediados del siglo XX, son celdas de combustible de alta temperatura de operación, entre \SI{600}{\celsius} y \SI{700}{\celsius}. Su electrolito esta compuesto de carbonatos fundidos de litio y sodio ($Li_2CO_3$ y $Na_2CO_3$) estabilizados por una matriz de fibras de alúmina ($Al_2O_3$). Suelen tener ánodos de níquel y cátodos de óxido de níquel.\\

Estas celdas pueden operar con una amplia variedad de combustibles, y, por su alta temperatura, no son tan susceptibles a contaminación por $CO$ o $CO_2$. Además, a diferencia del resto de las tecnologías, no son necesarios materiales catalizadores en los electrodos, ya que la combinación del níquel y las altas temperaturas proveen suficiente actividad electroquímica. Sin embargo, estas temperaturas generan problemas con los distintos materiales, reduciendo la vida útil de las celdas. Además tienen un electrolito altamente corrosivo y en estado líquido.\textsuperscript{\cite{FC-FundAndAppl}\cite{FCHandbook}}\\

\paragraph{Celda de Combustible de Óxido Sólido (SOFC)}

Las SOFC son celdas que llevan en continuo desarrollo desde mediados del siglo XX, y como indica su nombre, poseen un electrolito compuesto por un óxido en estado sólido, generalmente dióxido de zirconio ($ZrO_2$) o dióxido de cerio ($CeO_2$). Operan en rangos de temperatura muy elevados, de entre \SI{600}{\celsius} y \SI{1000}{\celsius}.\\

Estas celdas tienen la ventaja de tener un electrolito sólido, frenando la corrosión y permitiendo la fabricación en distintas geometrías. Además, todos sus materiales son de costo moderado. Como clara desventaja se encuentra la alta temperatura de operación, que trae problemas similares a los de las MCFC.\textsuperscript{\cite{FC-FundAndAppl}\cite{FCHandbook}}\\

\paragraph{Celda de Combustible de Ácido Fosofórico (PAFC)}

Las PAFC utilizan ácido fosfórico ($H_3PO_4$) con concetración de 100 \% estabilizado por una matriz basada en carburo de silicio ($SiC$) como electrolito, y operan en un rango de temperaturas entre \SI{150}{\celsius} y \SI{220}{\celsius}. Estas celdas son relativamente modernas y se destacan por su alta potencia, pudiendo llegar hasta \SI{20}{\mega\watt}, suficiente para una planta de generación intermedia.\\

Estas celdas son poco sensibles a contaminación de $CO$ y $CO_2$, y su baja temperatura de operación permite el uso de materiales comunes para su construcción. Sin enmbargo, su uso de ácido como electrolito requiere materiales más resistentes para sus electrodos.\textsuperscript{\cite{FC-FundAndAppl}\cite{FCHandbook}}\\

\subsubsection{Modelo Eléctrico de las PEMFC}

Las celdas del tipo PEM, como se describió en la anterior sección, son celdas de combustible de baja temperatura, con un electrolito sólido compuesto por una membrana de intercambio protónico. Para este trabajo se eligió este tipo de celdas por su extensivo desarrollo, fácil disponibilidad, bajo precio comparado con otras tecnologías, además de las ventajas ya mencionadas en la sección \ref{tipos_celdas}.\\

Entonces, debemos obtener un modelo eléctrico que caracterice a un stack de celdas tipo PEM, pudiendo luego implementar este modelo (en forma de una ecuación y curva tensión-corriente) en una simulación por computadora para evaluar el comportamiento del sistema completo.\\

Para comenzar, se debe encontrar una forma de cuantificar la energía química de las reacciones redox que ocurren dentro de la celda, pero esto no es tan sencillo como parece. Con este fin se utiliza el concepto de la \textit{energía libre de Gibbs}, que se podría definir como \quotes{la energía disponible para realizar trabajo externo} (en nuestro caso, el \quotes{trabajo externo} es mover los electrones por el circuito externo). Se define la \textit{energía libre de Gibbs de formación} $G_f$ como la energía de Gibbs tomando la energía cero a las condiciones normales de presión y temperatura.\\

Evidentemente, la energía entregada por la reacción es entonces la diferencia entre la energía $G_f$ de los productos y la energía $G_f$ de los reactivos, que por cuestiones de conveniencia se refieren a la energía por mol de producto y reactivo, indicado por una raya sobre la letra minúscula ($\bar{g_f}$).

\begin{equation}\label{delta_gibbs}
    \Delta\bar{g_f} = \bar{g_f}_{productos} - \bar{g_f}_{reactivos}
\end{equation}

Entonces, teniendo en cuenta la reacción redox de la ecuación \ref{redox_celda}, donde el producto es un mol de $H_2O$ y los reactivos son un mol de $H_2$ y medio mol de $O_2$, para nuestro caso la ecuación anterior resulta

\begin{equation}\label{delta_gibbs_celda}
    \Delta\bar{g_f} = \bar{g_f}_{(H_2O)} - \bar{g_f}_{(H_2)} - \frac{1}{2}\bar{g_f}_{(O_2)}
\end{equation}

Ahora, teniendo en cuenta que el trabajo eléctrico realizado es el producto de la carga por la tensión ($W_E=Q\cdot E$), y considerando un proceso sin irreversibilidades y con combustible y comburente puro, se puede decir entonces que el trabajo eléctrico es aproximadamente igual a la energía química entregada por la reacción de la celda, es decir que $W_E = \Delta\bar{g_f}$.\\

Lo que hace falta, entonces, es obtener la cantidad de carga que circula a través del circuito externo por cada mol de agua que se produce. Como se puede ver en las dos reacciones parciales de las ecuaciones \ref{redox_anodo} y \ref{redox_catodo}, por cada mol de $H_2O$ que se obtiene, dos átomos de hidrógeno pierden su electrón, y en consecuencia, dos electrones circulan a través de la carga. Entonces, si $e$ es la carga de un electrón (\SI{1.602e-19}{\coulomb}) y $N$ es el número de Avogadro (\num{6.022e-23}) que indica la cantidad de partículas en un mol, la carga por cada mol es

\begin{equation}\label{carga_mol}
    Q=-2\cdot Ne=-2\cdot F=\SI{192970}{\coulomb}
\end{equation}

Donde $F$ es la constante de Faraday, que indica la carga de un mol de electrones.\\

Reemplazando la ecuación \ref{carga_mol} en la expresión del trabajo eléctrico (recordando que es equivalente a $\Delta\bar{g_f}$), se obtiene la siguiente expresión de energía obtenida por mol de producto.

\begin{equation}\label{trabajo_elec}
    W_E=\Delta\bar{g_f}=-2F\cdot E
\end{equation}

Entonces, si despejamos la tensión de circuito abierto $E$ (es decir corriente nula) de la ecuación anterior, podemos obtener una expresión para esta tensión en función de la energía de Gibbs de formación de la reacción, que para una temperatura de \SI{80}{\celsius} de una celda tipo PEM típica es de \SI{-228.2}{\kilo\joule\per\mole}.\textsuperscript{\cite{FCSysExplained}}

\begin{equation}\label{tension_vacio}
    \boxed{E=-\frac{\Delta\bar{g_f}}{2F}=\SI{1.183}{\volt}}
\end{equation}

Con esta ecuación, por lo tanto, se puede obtener la {\Medium tensión de circuito abierto de celda} \textit{teórica} de una celda de combustible cualquiera; pero se debe tener en cuenta que este valor es ideal, y no tiene en cuenta múltiples factores que reducen la eficiencia (y la tensión de circuito abierto) del dispositivo: no es posible utilizar el 100 \% del combustible disponible, algunas dinámicas de las reacciones utilizan parte de la energía química generada, entre otros. Además, en este desarrollo no se consideró la variación de la energía libre de Gibbs con la presión y concentracion de gases.\\

\paragraph{Modelo Tensión-Corriente}

Sin embargo, esto no es suficiente para un análisis eléctrico completo del dispositivo. Ahora se deben describir las distintas partes de una curva típica de tensión-corriente de una celda de combustible de baja presión y temperatura (como las PEMFC), y al mismo tiempo presentar las ecuaciones que la describen para poder obtener el modelo eléctrico completo que se busca. Se puede ver esta curva típica en la figura \ref{V-I_celda}.\\

\begin{figure}[h]
    \centering
    \includegraphics[scale=0.28]{Imagenes/Curva V-I Celda.png}
    \caption{Curva de tensión vs. corriente típica de una celda de combustible tipo PEM, con sus tres regiones marcadas.}
    \label{V-I_celda}
\end{figure}

En esta curva se pueden señalar tres regiones de pérdidas bien marcadas: la región de {\Medium pérdidas de activación} cerca de corriente nula, seguida por la región de {\Medium pérdidas óhmicas}, y finalmente, acercándose a la máxima corriente, la región de {\Medium pérdidas de concentración}. Estas pérdidas se dan por algunas irreversibilidades de las reacciones que ocurren en la celda, que la alejan de su comportamiento ideal. A continuación se detallan estos componentes y se obtienen sus ecuaciones correspondientes.\\

\subparagraph{Región de Pérdidas de Activación ( I )}

Como se puede ver, en la primera región hay una rápida caída de tensión de características no lineales. Esto ocurre por las llamadas \textit{pérdidas de activación}, que se generan por la lenta velocidad de reacción en las superficies de los electrodos para bajas densidades de corriente. Una porción de la tensión generada se pierde al generar la reacción electroquímica, que transfiere los electrones desde o hacia los electrodos.\\

La ecuación asociada este comportamiento, formulada empíricamente por el químico suiso Julius Tafel en 1905, es una ecuación que es describe la caída de tensión en un electrodo para una gran variedad de reacciones, incluida la rección redox de agua que nos interesa. La ecuación de Tafel relaciona la caída de tensión en un electrodo $\Delta V_{act}$ con la densidad de corriente $i$ que circula a través del mismo mediante una forma logarítmica.
    
\begin{equation}\label{perd_act}
    \Delta V_{act}=A\cdot \ln\left(\frac{i}{i_0}\right)
\end{equation}

La constante $i_0$ (llamada \textit{densidad de corriente de intercambio}) se puede considerar como la densidad de corriente para la cual la tensión de celda se comienza a alejar de la ideal de la ecuación \ref{tension_vacio}, y su valor aumenta mientras más rápida sea la reacción. En contraste, la constante $A$ que multiplica al logaritmo es mayor para una reacción electroquímica lenta.\\

En el caso particular del hidrógeno como combustible, las pérdidas se concentran casi únicamente en el ánodo (donde ocurre la oxidación), con la densidad $i_0$ del ánodo generalmente mas de \num{10000} veces mayor a la del cátodo, por lo que generalmente las pérdidas de activación de este último se pueden despreciar, teniendo en cuenta únicamente las del ánodo.\\

\subparagraph{Región de Pérdidas Óhmicas ( II )}

Esta región es la que abarca el mayor rango de corrientes de celda, además de ser la más simple de modelar y entender. En este caso, las pérdidas se dan simplemente por la resistencia eléctrica al paso de corriente de ambos electrodos y la resistencia al paso de iones del electrolito, y por lo tanto, la caída de tension $\Delta V_{ohm}$ esta relacionada linealmente con la densidad de corriente $i$ mediante la Ley de Ohm.

\begin{equation}\label{perd_ohm}
    \Delta V_{ohm}=i\cdot r
\end{equation}

Donde $r$ debe ser la resistencia por unidad de área (\unit{\ohm\metre\squared}) si se trabaja con $i$ como densidad de corriente (\unit{\ampere\per\metre\squared}).\\

\subparagraph{Región de Pérdidas de Concentración ( III )}
    
Esta última región de pérdidas viene dada, como dice su nombre, por la reducción de la concentración de combustible y comburente en el ánodo y cátodo respectivamente, condición que se ve exacerbada al trabajar con corrientes y cargas muy elevadas. Esta reducción en concentración se traduce a una reducción de la tensión de celda $\Delta V_{conc}$.\\

En general, el consenso es que no existe una única ecuación analítica que sea capaz de describir este comportamiento para cualquier caso. Entonces, hoy en día es muy común el uso de una ecuación de bases empíricas que, con la correcta elección de constantes, se ajusta muy bien al comportamiento real observado experimentalmente, y relaciona $\Delta V_{conc}$  exponencialmente con la densidad de corriente $i$.

\begin{equation}\label{perd_conc}
    \Delta V_{conc}=m\cdot e^{ni}
\end{equation}

Donde las constantes $m$ y $n$ suelen estar alrededor de \SI{3e-5}{\volt} y \SI{8e-3}{\cm\squared\per\milli\ampere} respectivamente.\\

Habiendo obtenido la ecuacion para la tensión irreversible (ecuación \ref{tension_vacio}) y las ecuaciones de cada una de las tres regiones (ecuaciones \ref{perd_act}, \ref{perd_ohm} y \ref{perd_conc}), se pueden combinar todas en una única expresión que modela la tensión de una celda para cualquier densidad de corriente:

\begin{equation}
    \begin{aligned}
        V_{celda} & = E - \Delta V_{act} - \Delta V_{ohm} - \Delta V_{conc}\\
                  & = E - A\cdot \ln\left(\frac{i}{i_0}\right) - i\cdot r - m\cdot e^{ni}
    \end{aligned}
\end{equation}

Sin embargo, todavía se pueden realizar algunas simplificaciones. Para la ecuación \ref{perd_act}, que expresa la caída de tension por activación, la densidad de corriente de intercambio $i_0$ es muy baja, mucho menor a la densidad de corriente $i$, por lo que esta ecuación se puede modificar de la siguiente manera:

\begin{equation*}
    \Delta V_{act} = A\cdot \ln(i) - A\cdot \ln(i_0)
\end{equation*}

Como el último término solo depende de $i_0$, que es un valor constante, se lo puede agrupar con la tensión irreversible $E$, para obtener una tensión de circuito abierto real y reversible $E_{oc}$.

\begin{equation}
    E_{oc} = E + A\cdot\ln(i_0)
\end{equation}

Vale aclarar que, al ser la densidad de corriente de intercambio una magnitud muy chica, al calcular su logaritmo natural se obtiene un número negativo, por lo que la tensión de circuito abierto reversible $E_{oc}$ resulta, como es esperable, menor la la tensión irreversible $E$. Entonces, la expresión final que describe la relación tensión vs. corriente de una celda de combustible se muestra en la siguiente ecuación.

\begin{equation}\label{v_celda}
    V_{celda}(i) = E_{oc} - A\cdot\ln(i) - i\cdot r - m\cdot e^{ni}
\end{equation}

Para obtener la tensión de una pila, solo es necesario multiplicar la tensión $V_{celda}$ por la cantidad de celdas $N$ del stack.

\begin{equation}\label{v_stack}
    \boxed{
    \highlight{V_{stack}(i) = NV_{celda}(i) = N\cdot (E_{oc} - A\cdot\ln(i) - i\cdot r - m\cdot e^{ni})}
    }
\end{equation}

\subsubsection{Emulador de Celdas de Combustible}

Esta plataforma se basa en la utilización de un modelo comercial particular de pila de combustible: el módulo H-300 de la serie H de pilas de combustible de Horizon Fuel Cell Technologies de la figura \ref{H300}. Esta es una pila de combustible de \SI[]{300}[]{\watt} enfriada por aire del tipo PEMFC, qur consiste en un stack de \num{60} celdas. Su desempeño nominal es de \SI[]{36}[]{\volt} de tensión a \SI[]{9}[]{\ampere} de corriente y tiene una tensión de circuito abierto de aproximadamente \SI[]{60}[]{\volt}.\textsuperscript{\cite{HSeriesBrochure}}\\

\begin{figure}[h]
    \centering
    \includegraphics[scale=0.4]{Imagenes/Horizon H-300.png}
    \caption{Pila de combustible tipo PEMFC, modelo Horizon Fuell Cell Technologies H-300.}
    \label{H300}
\end{figure}

Sin embargo, dado que este módulo no está presente en el laboratorio, como reemplazo se utiliza un {\Medium módulo de emulación de pilas de combustible}. Este módulo permite la reproducción de una curva tensión-corriente de una pila en condiciones de trabajo controladas.\\

Este módulo consiste de un convertidor CC-CC conmutado de tipo reductor, (que se explicará mas adelante) que impone una tensión de salida en función de la corriente suministrada a través de un lazo de control. El modelo que utiliza para obtener la tensión de salida en función de la corriente es el que se se obtuvo en la ecuación \ref{v_stack} y en el gráfico de la figura \ref{V-I_celda}. El módulo se puede ajustar a distintos modelos de celdas mediante la variación de las constantes $N$, $E_{oc}$, $A$, $r$, $m$ y $n$. Los valores de la curva se almacenan en una tabla de \textit{look-up} implementada en un FPGA.\\

Adicionalmente a la característica tensión-corriente de la celda, este módulo permite simular el filtro pasabajos propio de la pila que se ve en el diagrama de la plataforma de la figura \ref{diag_plataforma}. Este filtro cumple la función de proteger a las celdas y evitar deterioro de las mismas mediante una reducción del rizado de corriente que puede generar la conmutación del convertidor.\textsuperscript{\cite{Argencon2018}}\\