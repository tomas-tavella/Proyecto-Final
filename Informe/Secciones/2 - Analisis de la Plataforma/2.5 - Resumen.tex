\subsection{Resumen}

En este capítulo se planteó un análisis exhaustivo de cada uno de los componentes que integran la plataforma: las celdas de combustible, el convertidor CC-CC conmutado, el sistema de control, y finalmente la carga electrónica variable. Se trató cada uno de estos temas en profundidad, pero poniendo un énfasis particular en los primeros dos, que son el foco principal de este proyecto.\\

Para las celdas de combustible, se partió de una explicación de su principio básico de funcionamiento, para luego pasar a un listado y breve explicación de cada uno de los distintos tipos de celdas disponibles. Habiendo establecido las PEMFC como el tipo de celda a utilizar, se derivó la ecuación para el modelo eléctrico de una PEMFC, y se cerró la sección con una breve descripción del emulador de celdas.\\

Luego se pasó al convertidor CC-CC conmutado, encargado de convertir la tensión variable de la celda de combustible a la tensión continua fija de la salida. Se partió de algunas nociones básicas de convertidores de continua-continua, para luego describir el convertidor CC-CC conmutado más básico: el reductor. En base al circuito reductor, se fue desarrollando y agregando componentes hasta llegar al convertidor de puente completo que utiliza la plataforma.\\

Se dedicaron unas breves páginas a la descripción de las funciones más importantes del sistema embebido que conforma nuestro sistema de control, para luego cerrar el capítulo con unos párrafos dedicados a la carga electrónica de laboratorio que se va a utilizar para simular las condiciones de carga.\\

Estas descripciones generales que se llevaron a cabo durante el capítulo van a ser de utilidad para comprender los criterios de selección de componentes que se utilizan en el siguiente capítulo.\\