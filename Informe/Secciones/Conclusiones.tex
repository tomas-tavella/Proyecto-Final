\section{Conclusiones}\label{concluisiones}
\AddToShipoutPictureBG*{\includegraphics[width=\paperwidth,height=\paperheight]{Imagenes/Fondo Capitulo 6.pdf}}
\thispagestyle{plain}

%\vspace{0.3cm}
\divider

Este trabajo consistió en el desarrollo completo de una plataforma electrónica integrada, con el propósito de evaluar sistemas híbridos basados en pilas de combustible, a lo largo de nueve meses. Este desarrollo se llevó a cabo en múltiples etapas, todas explicadas con lujo de detalle en este informe.\\

Primero que nada, se comenzó por una recopilación y lectura de bibliografía, incluyendo libros de texto, notas de aplicación y otros proyectos del laboratorio, para empezar a entender el funcionamiento de los distintos componentes que conformaban la plataforma en estudio.\\

Aquí entonces comienza el desarrollo propiamente dicho de la plataforma. Se empezó por la realización de simulaciones numéricas del funcionamiento del convertidor simplificado, para luego ir uno por uno diseñando todos los circuitos auxiliares que conforman el sistema, y seleccionando los componentes apropiados a través catálogos online de partes electrónicas.\\

Con todos los circuitos diseñados, se continuó con la implementación de los mismos, para traspasarlos del plano de las ideas hacia la realidad. Se desarrolló, entonces, el diseño de una compleja placa de circuito impreso que agrupa todos los componentes y circuitos, y los integra para formar el hardware de la plataforma.\\

Una vez hecho esto, se enviaron los diseños a un fabricante de circuitos impresos, y después de algún tiempo de espera, arribaron las placas fabricadas al laboratorio. Con todos los componentes necesarios disponibles, se procedió a soldar todos los componentes, en un proceso que duró alrededor de un mes y culminó con la finalización de la plataforma.\\

Finalmente, con el sistema completo, se inició una breve serie de ensayos con el objetivo de asegurar que los componentes de la plataforma funcionaran dentro de los parámetros esperados. Este se puede considerar un objetivo cumplido, ya que el sistema se comportó adecuadamente bajo la carga y el escrutinio.\\

Sin embargo, se presentaron durante todo el proceso una serie de contratiempos, entre los cuales se destaca el diseño e implementación de la placa adaptadora para los transistores, que alargaron el proceso y forzaron a acortar la duración del proyecto. Esto resultó en la exclusión de la parte dedicada a la  programación de algoritmos de control del proyecto.\\

En cualquier caso, con el diseño y la implementación del hardware completos, se considera cumplido el principal objetivo del proyecto, que apuntaba al desarrollo de una plataforma experimental para la evaluación de sistemas de pilas de combustible. Además, el estado final en el que este proyecto deja la plataforma, abre la puerta a la posibilidad de una continuación en su desarrollo, en particular en el área de control automático y desarrollo de firmware.\\

