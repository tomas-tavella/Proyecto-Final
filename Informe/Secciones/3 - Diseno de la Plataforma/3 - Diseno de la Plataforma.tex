\section{Diseño de la Plataforma} \label{diseño}
\thispagestyle{plain}

\vspace{0.5cm}

\Large\scshape
\begin{center}
    \textrm{Diseño de los circuitos principales y auxiliares de la plataforma}
\end{center}
\normalfont
%\normalsize

\divider

Si bien con el análisis del anterior capítulo se pudo conseguir un panorama general del funcionamiento de la plataforma, se presentan otras complejidades a la hora de plasmarlo en un sistema real: se requieren múltiples circuitos auxiliares además de los bloques principales (por ejemplo circuitos de adquisición de señales); aparecen consideraciones de diseño que no existen en el plano teórico; entre otras cuestiones. Este capítulo está dedicado al diseño real de la plataforma completa para luego implementar en una placa de circuito impreso o PCB, teniendo en cuenta estas complicaciones.\\

En la siguiente figura se muestra un diagrama detallado de la plataforma, dónde se presentan todos los distintos bloques funcionales, incluyendo los bloques auxiliares que no se trataron en el análisis del anterior capítulo.\\

\begin{figure}[h]
    \centering
    \includegraphics[scale=0.2]{Imagenes/Plataforma Detallada.pdf}
    \caption{Diagrama detallado de la plataforma de evaluación, incluyendo los distintos circuitos auxiliares (Placeholder).}
    \label{diag_detallado}
\end{figure}

Cada uno de estos seis bloques cumplen una función específica que se detalla a continuación:\\

\begin{itemize}
    \item {\SemiBold Convertidor CC-CC Conmutado:} Este es el convertidor de tipo puente completo que se trató en el capítulo anterior. En este capítulo se va a realizar el dimensionamiento de todos sus componentes teniendo en cuenta sus especificaciones. 
    \item {\SemiBold Circuito \textit{Driver}:} Este circuito se encarga de entregar la corriente y tensión necesaria para disparar los transistores de potencia y conmutarlos correctamente.
    \item {\SemiBold Sistema de Medición:} Este bloque contiene todos los circuitos y componentes necesarios para realizar las mediciones de todos los parámetros de interés de la plataforma. Esto incluye, además de sensores, los circuitos de acondicionamiento de señal donde se requieran.
    \item {\SemiBold Etapa de Aislación:} Esta etapa se encarga de generar una barrera de aislación eléctrica entre los componentes de potencia y los componentes de señal del circuito.
    \item {\SemiBold Sistema de Control:} Este es el bloque de control que se explicó en el anterior capítulo. Obtiene información de distintos parámetros por medio del sistema de medición, y ejerce la acción de control disparando las llaves mediante el driver.
    \item {\SemiBold Circuito de Alimentación:} Es el circuito que se encarga de proveer las corrientes y tensiones necesarias para los componentes que requieren alguna alimentación externa para funcionar (por ejemplo el controlador digital de señales).\\
\end{itemize}

A lo largo de este capítulo se va a tratar uno por uno el diseño de los circuitos que componen a cada uno de los bloques, utilizando múltiples diseños como referencia (ya sean de otros trabajos de investigación o diseños sugeridos de los propios fabricantes). Se van a eligir y dimensionanar los componentes que forman parte de ellos, hasta obtener un esquemático circuital detallado de la plataforma experimental de evaluación completa.\\

Pero antes de comenzar con el primer bloque, se van a plantear algunas consideraciones y criterios generales que se van a utilizar para la selección de todos los componentes y diseño de todos circuitos de la plataforma.\\

\subsection{Consideraciones Generales}

\subsubsection{Aislación de Tierras}

En toda la plataforma se va a trabajar con tres puestas a tierra distintas y aisladas entre sí: $GND_1$ es la tierra del primario del convertidor, $GND_2$ es la tierra del secundario del convertidor, y $GND_D$ es la tierra de las partes de señal y digitales, como los sensores y el DSC.\\

Esto, si bien agrega una mayor complejidad al diseño, es ventajoso por múltiples razones. Primero, evita la generación de interferencia de modo común entre las tierras del convertidor ($GND_1$ y $GND_2$) que manejan altas corrientes y por lo tanto son más ruidosas; y la tierra de señal $GND_D$ de más bajas corrientes que es más sensible al ruido. Además, dadas las altas corrientes del convertidor, esta separación permite la protección de los circuitos de señal ante picos de corriente y tensión inesperados en la parte de potencia.\\

Es por estas razones que además de la tierra, también los circuitos de señal y potencia se encuentran separados por la etapa de aislación entre potencia y señal. Adicionalmente, las fuentes de alimentación externas se encuentran separadas para los componentes de potencia y señal, manteniendo la aislación deseada.\\

\subsubsection{Selección de Componentes}

En líneas generales, a la hora de elegir un circuito para el diseño de los distintos bloques, si es posible se trata de elegir una solución más integrada (es decir utilizar un circuito integrado que haga esta tarea en vez de diseñar un circuito discreto). Esto simplifica los circuitos y disminuye la cantidad de componentes necesarios a la hora de implementarlos. Además, al estar toda la solución integrada, el rendimiento es más predecible y se encuentra acotado a los parámetros dados por el fabricante del circuito integrado.\\

En todos los casos, se utilizan como guía para el diseño de todas las partes los parámetros de rendimiento y las recomendaciones de diseño especificadas en las hojas de datos  y notas de aplicación de los fabricantes de cada circuito integrado.\\

\subsubsection{Herramientas de Software}

\paragraph{Software EDA}

Para realizar el diseño de todos los esquemas circuitales del sistema, y luego plasmarlos a una placa de circuito impreso se debe utilizar un herramienta de automatización de diseño electrónico o EDA (del inglés \textit{Electronic Design Automation}). Existe una gran variedad de programas que cumplen este propósito, estando entre los más conocidos el \textit{Altium Designer} de \textit{Altium}, el \textit{EAGLE} de \textit{Autodesk}, el \textit{KiCad} y el \textit{Proteus Design Suite} de \textit{Labcenter Electronics}.\\

Para este proyecto se eligió utilizar la plataforma {\Medium KiCad} (que se encuentra en la versión 6.0.7 al momento de escribir este informe), una suite de software libre, gratuita y de código abierto que incluye todas la funcionalidades necesarias para el diseño electrónico. Cuenta con herramientas de captura de esquemático, diseño de PCB, simulación mediante SPICE o Ngspice, visualización de archivos de fabricación y cálculos de diseño de PCB.\\

\begin{figure}[h]
    \centering
    \includegraphics[scale=0.6]{Imagenes/KiCad.pdf}
    \caption{Logotipo de la plataforma KiCad EDA.}
    \label{logo_kicad}
\end{figure}

El programa también cuenta con una extensa biblioteca de componentes y \textit{footprints} (son las \quotes{huellas} de los componentes en en el circuito impreso) y la capacidad de crear o importar bilbiotecas. Además tiene la capacidad de generar archivos de fabricación, modelos tridimensionales de la PCB y una \textit{bill of materials} (lista de componentes).\\

\paragraph{Software de Simulación}

Para todo lo que se refiere a la simualción de la plataforma; más particularmente las simulaciones del funcionamiento del convertidor CC-CC para su comprensión, estudio, diseño y dimensionamiento; se utilizó la herramienta {\Medium\textit{Simulink}} dentro de la suite de software de \textit{MATLAB-Simulink}.\\

\begin{figure}[h]
    \centering
    \includegraphics[scale=0.08]{Imagenes/Simulink.png}
    \caption{Logotipo de la plataforma de simulación Simulink.}
    \label{logo_simulink}
\end{figure}

Específicamente, para simulaciones circuitales se hizo uso de el paquete \textit{Simscape Electrical} dentro de Simulink, que permite trabajar con tensiones y corrientes, a diferencia de las herramientas estándar que trabajan con diagramas de bloques.\\

\paragraph{Otras Herramientas}

Adicionalmente, para llevar un control de versiones completo del diseño de la plataforma sobre el que se trabaja, además de mantener un historial completo de todos los cambios, se trabajó con la herramienta de software de control de versiones {\Medium\textit{Git}}.\\

\begin{figure}[h]
    \centering
    \includegraphics[scale=0.6]{Imagenes/Git.pdf}
    \caption{Logotipo del software de control de versiones Git.}
    \label{logo_git}
\end{figure}

Con este software se crea un \textit{repositorio} donde se almacenan los archivos que se quiere controlar, manteniendo un control de la historia de cada uno de los archivos del repositorio. Para mantener los archivos sincronizados entre varias computadoras y mantener copias de seguridad, se utiliza adicionalmente la plataforma web {\Medium\textit{GitHub}} para hostear el repositorio en la nube, manteniendo una copia segura que se puede copiar a cualquier computadora.\\

\newpage

\subsection{Convertidor CC-CC Conmutado}

\newpage

\subsection{Circuito Driver}

Como se explicó más arriba, para excitar un transistor MOSFET y encenderlo, es necesario mantener una tensión $V_{GS}$ entre gate y source mayor a una tensión umbral dependiente del modelo. En nuestro caso, esta tensión umbral del IRFP150N es de \SI[]{4}[]{\volt}, como se ve en la tabla \ref{tabla:IRFP150}. Entonces, se debe diseñar algún circuito que sea capaz de proveer estos pulsos de tensión al gate de cada transistor, entregando también la corriente necesaria para cargar y descargar sus capacitancias de gate suficientemente rápido (llamadas corrientes de \textit{source} y \textit{sink}).\\

Este es el llamado {\Medium circuito \textit{driver}} o {\Medium circuito de excitación} y debe existir uno para cada uno de los cuatro transistores del puente. Ahora debemos establecer algunos requerimientos que debe cumplir el circuito:\\

\begin{itemize}
    \item Tensión de operación mayor a \SI[]{100}[]{\volt}, por encima de la máxima tensión de la pila de combustible.
    \item Tiempos de encendido y apagado mucho menores al período $T_s$ de \SI[]{50}[]{\micro\second} de la excitación.
    \item Corrientes de sink y source mayores a \SI[]{2}[]{\ampere} para cargar rápidamente las capacitancias de los transistores, calculado según la nota de aplicación de \cite{SinkSourceCurrent}.
    \item Se busca utilizar una solución integrada, ya que suelen ser más compactas y sencillas.
    \item Es deseable el uso de componentes de montaje superficial o SMD.\\
\end{itemize}

Con estos datos vamos a seleccionar y diseñar un circuito de excitación y explicar brevemente el funcionamiento de todas sus partes.\\

\subsubsection{Selección y Diseño}

Existen diversos tipos de soluciones integradas para circuitos de excitación de transistores MOSFET. Se pueden encontrar circuitos de uno o múltiples canales; existen circuitos que incluyen una aislación entre las entradas y salidas; entre otras funcionalidades. También se consiguen con distintas funciones de seguridad y protección, como el \textit{dead-time}, que permite forzar un tiempo fijo entre la activación de dos transistores de la misma rama, evitando situaciones de cortocircuito; y el \textit{undervoltage lockout} (UVLO), que evita daños por condiciones de baja tensión.\\

Entre todas las opciones, originalmente se había decidido por el modelo UCC21540 de Texas Instruments, un driver de doble canal, con aislación incluida, funcionalidades de dead-time y UVLO, alta capacidad de corriente y un encapsulado SMD de tipo SOIC-16.\\

Sin embargo, este dispositivo no se pudo obtener por falta de disponibilidad, por lo que se tuvo que buscar una alternativa de características similares que esté en disponibilidad. Se terminó decidiendo por el integrado {\Medium 2ED21834-S06J de Infineon Technologies}, cuyas especificaciones básicas se muestran a continuación.\\

\setlength{\tabcolsep}{7pt}
\renewcommand{\arraystretch}{1.5}
\begin{table}[h]
\begin{center}
    \begin{tabular}{llrrrr}
    {\SemiBold Fabricante} & {\SemiBold Modelo} & $\mathbf{V_S}$ [\unit{\volt}] & $\mathbf{I_{OH}/\mathbf{I_{OL}}}$ [\unit{\ampere}] & $\mathbf{t_{on}}/\mathbf{t_{off}}$ [\unit{\nano\second}] & $\mathbf{V_{cc}}$ [\unit{\volt}]\\
    \hline
    \makecell[l]{Infineon \\ Technologies} & 2ED21834-S06J & \num{650} & \num{2.5} & \num{200} & \num{10}-\num{20}
    \end{tabular}
    \caption{Especificaciones del driver modelo 2ED21834-S06J de Infineon Technologies.\textsuperscript{\cite{DatasheetDriver}}}
    \label{tabla:driver}
\end{center}
\end{table}

Donde $V_S$ es la máxima tensión común de operación, $I_{OH}$ e $I_{OL}$ son las corrientes máximas de source y sink, $t_{on}$ y $t_{off}$ son los tiempos de encendido y apagado, y $V_{cc}$ es el rango de tensiones de alimentación.\\ 

El 2ED21834-S06J es un driver de doble canal para medios puentes de transistores de tipo MOSFET e IGBT, con diodo y resistencia de bootstrap incluidos además de funcionalidad de dead-time y UVLO para circuitos del lado bajo y alto, todo contenido en un encapsulado SMD de catorce pines del tipo DSO-14 (figura \ref{encapsulado_driver}). Sus corrientes sink y source de \SI[]{2.5}[]{\ampere} superan la corriente necesaria calculada para los IRFP150N de la tabla \ref{tabla:IRFP150}, su tensión de operación se encuentra cómodamente por encima de la tensión de operación del primario del convertidor, además de tener muy bajos tiempos de conmutación.\\

\begin{figure}[h]
    \centering
    \includegraphics[scale=0.07]{Imagenes/Driver DSO-14.png}
    \caption{Driver 2ED21834-S06J con su encapsulado SMD tipo DSO-14.}
    \label{encapsulado_driver}
\end{figure}

En nuestro caso, se deben utilizar dos de estos dispositivos, uno para cada columna del puente completo. Vamos a utilizar la función de dead-time, configurable mediante una resistencia conectada al pin DT, para proteger contra posibles cortocircuitos causados por la activación errónea de ambos transistores de una columna simultáneamente (\textit{shoot-through}). El resto de la conexión de componentes del driver se realizó de acuerdo a las recomendaciones del fabricante encontradas en la hoja de datos \cite{DatasheetDriver}, que se puede ver en la figura \ref{circuito_driver}.\\

\begin{figure}[h]
    \centering
    \includegraphics[scale=0.95]{Imagenes/Circuito Driver.png}
    \caption{Circuito de conexión del driver 2ED21834-S06J. El circuito del driver para la otra columna es idéntico.}
    \label{circuito_driver}
\end{figure}

Aquí se puede ver el driver, indicado por la referencia U12, al que le llegan las señales de comando PWM a sus pines HIN y /LIN para el transistor del lado alto y bajo de la columna respectivamente (al estar negada  la entrada para el transistor bajo, la señal que le llega debe estar invertida). Luego, conectado entre el pin DT y tierra se encuentra la resistencia de dead-time, que cuyo valor define el dead-time o tiempo muerto $t_{DT}$. En la salida, se conecta a los pines HO (alto) y LO (bajo) una resistencia limitadora en paralelo con un diodo que permite la descarga de las capacitancias de los transistores, y entre los pines VB y VS se coloca el capacitor que completa el circuito de bootstrap, que se explicará mas adelante. En lo que hace referencia a conexiones a tierra, este circuito, al estar del lado primario del convertidor, se conecta a la referencia $GND_1$.\\

El dimensionamiento de todos estos componentes se va a tratar a continuación siguiendo las recomendaciones del fabricante disponibles en hojas de datos y notas.

\subsubsection{Dimensionamiento de Componentes}

\lipsum[1]\\

\lipsum[2]\\

\subsubsection{Esquema Interno del Dispositivo}

\lipsum[1]\\

\begin{figure}[h]
    \centering
    \includegraphics[scale=0.25]{Imagenes/Esquema Interno Driver.png}
    \caption{Diagrama de bloques interno del driver 2ED21834-S06J de Infineon Technologies.}
    \label{interno_driver}
\end{figure}

\lipsum[2]\\

\lipsum[3]\\

\lipsum[4]\\

\newpage

\subsection{Sistema de Medición}

\newpage

\subsection{Etapa de Aislación}

\newpage

\subsection{Sistema de Control}

Ahora, se deben diseñar los circuitos auxiliares del sistema de control que rodean al controlador digital de señales. Estos son mayormente circuitos muy sencillos, ya que la gran parte de los circuitos que implementan las funcionalidades de periféricos se encuentran incluidos en el paquete de evaluación (controlCARD, de la figura \ref{ControlCARD}) del DSC.\\

Para el diseño de estos circuitos de la plataforma correspondiente al sistema de control, se utilizó como referencia una placa adaptadora para la controlCARD del controlador TMS320F28335 que se encuentra en el laboratorio. Esta placa es utilizada para otros proyecto de control del laboratorio, y hace uso de otras funcionalidades adicionales del DSC, como el módulo dedicado para el uso de encoders. Se decidió por basar el sistema de control en este diseño, principalmente porque ya ha tenido extenso uso para control y es comprobado que funciona correctamente.\\

\subsubsection{Implementación de Periféricos}

El módulo de controlCARD en el que se encuentra el TMS320F28335 se conecta mediante la interfaz DIMM-100 de 100 pines. Algunos de estos pines son dedicados a la alimentación y conexión a tierra del dispositivo, interfaz JTAG (que trataremos ahora) y puertos de conversor analógico-digital, con el restante de los pines mapeados a distintos puertos de entrada-salida de propósito general (GPIOxx).\\

\begin{figure}[h]
    \centering
    \includegraphics[scale=0.5]{Imagenes/DIMM100.png}
    \caption{Socket de conexión de cien pines de tipo DIMM-100, donde se inserta la controlCARD.}
    \label{dimm100}
\end{figure}

Estos pines de GPIO pueden ser configurados mediante un registro del DSC para cambiar su funcionalidad: cada pin cuenta con un multiplexor, de manera que, por ejemplo, el pin GPIO 0 puede funcionar como pin de entrada/salida general, o bien como salida de la señal ePWM1A del módulo ePWM. Entonces, si observamos el \textit{pinout} de la controlCARD, podemos decidir cuáles pines del conector asignar para cada funcionalidad.\\

\paragraph{Salidas ePWM}

Como se mencionó en el capítulo \ref{analisis}, el TMS320F28335 cuenta con seis módulos ePWM, cada uno con dos salidas, A y B, es decir doce salidas totales. La controlCARD tiene pines asignados para cada una de estas salidas. En nuestro caso vamos a usar únicamente los primeros cuatro pines, correspondientes a las salidas ePWM1A, ePWM1B, ePWM2A y ePWM2B (Pines GPIO 0, 1, 2 y 3).\\

Estos pines se van a utilizar de manera que las dos salidas del módulo ePWM1 controlan la pata izquierda del puente, y las dos del módulo ePWM2 controlar la pata derecha. Las dos salidas de cada módulo se configuran para estar en contrafase, con un dead-time agregado entre estas señales de \SI[]{200}{\nano\second}. Luego, internamente se conecta la salida de sincronización SYNCO del primer módulo a la entrada de sincronización SYNCI del segundo módulo, y mediante el registro de fase del segundo módulo se controla su fase relativa a ePWM1, logrando una implementación del control de ciclo de trabajo mediante phase-shift que se explicó en la descripción del convertidor.\\

Las señales ePWM utilizadas, junto con las señales ePWM3A y ePWM4A (GPIO 4 y 6) del tercer módulo del DSC son conectadas también a una tira de pines para facilitar la medición de estas señales con un osciloscopio. Además, las señales del tercer módulo se pueden utilizar para implementar una funcionalidad adicional por fuera de los componentes de la plataforma.\\

\paragraph{Entradas de ADC}

Para la entrada de datos proveniente del sensor de efecto hall TMCS1100A4, se utiliza el módulo del conversor analógico digital de 12 bits y \SI[]{80}{\nano\second} de tiempo de conversión que se describió en el capítulo \ref{analisis}. Nuevamente, la controlCARD tiene pines asignados para cada una de las 16 (8 por cada canal A y B) entradas del conversor analógico-digital.\\

Para los datos del sensor se utiliza la primer entrada del canal A, ADCIN-A0, y se conectó una tira de pines a las tres primeras entradas del convertidor, hasta ADCIN-A2. Esto cumple el mismo propósito que la tira de pines para las señales ePWM.\\

\paragraph{Comunicación Serie}

Todos los controladores de la serie C2000 implementan cuatro protocolos de comunicación serie distintos: un bus {\Medium I\textsuperscript{2}C} (\textit{Inter-Integrated Circuit}), una interfaz {\Medium SPI} (\textit{Serial Peripheral Interface}), un módulo {\Medium SCI} (\textit{Serial Communications Interface}) más conocido como UART, y un módulo {\Medium McBSP} (\textit{Multichannel Buffered Serial Port}). El único protocolo que no se va a utilizar en esta plataforma es el de McBSP.\\

\subparagraph{I\textsuperscript{2}C}

El módulo I\textsuperscript{2}C, utilizado para transmitir los datos de tensión, corriente y potencia que entrega el LM5056A, se describió en detalle en el capítulo del sistema de medición. Las líneas de datos SDA y de reloj SCL estan multiplexadas a los pines GPIO 32 y 33 del controlador respectivamente. El cálculo de las resistencias de pull-up y el circuito de separación de la línea de datos ya se trató en el análisis detallado del protocolo.\\

Ambas líneas del bus también tienen salida a través de una tira de pines, para facilitar la evaluación del funcionamiento del bus, y conectar dispositivos adicionales, si se llegara a requerir.\\

\subparagraph{SCI (UART)}

El protocolo SCI, mejor conocido como UART (\textit{Universal Asynchronous Reciever-Transmitter}), es un bus de transmisión de datos en serie, asincrónico (sin señal de reloj para sincronizar), y full-duplex (capacidad de transmitir datos en ambas direcciones simultáneamente). Es un protocolo de velocidad de transmisión variable, y es común su utilización para la comunicación entre un controlador y dispositivos periféricos.\\

El TMS320F28335 cuenta con tres módulos independientes para comunicación por SCI (módulos A, B y C). En la plataforma se utilizó un único módulo para la transmisión de datos mediante el puerto USB. Esto se implementó mediante un integrado que convierte los datos que ingresan por el puerto a un formato para ser transmitido por UART. Las líneas de recepción SCIRX-A y transmisión SCITX-A del primer módulo corresponden a los puertos GPIO 28 y 29 respectivamente. Adicionalmente, al igual que el I\textsuperscript{2}C, se conectaron ambas líneas a una tira de pines.\\

\subparagraph{SPI}

La interfaz SPI es un protocolo de transmisión de datos en formato serie, sincrónico y full-duplex desarrollado por Motorola, y es generalmente utilizado para comunicaciones de corta distancia en sistemas embebidos. Es un bus multi-dispositivo y de esquema maestro/esclavo que cuenta con cuatro lineas: SPI-SIMO, para transmisión de datos del maestro hacia el esclavo; SPI-MISO, para transmisión de datos del esclavo hacia el maestro; SPI-CLK para transmitir la señal de reloj; y SPI-STE, que implementa un \textit{chip-select} para seleccionar el esclavo correcto.\\

En el controlador, este protocolo se implementa para conectar un módulo para lectura y escritura de tarjeta SD (se suelen vender módulos completos que incluyen la traducción de datos al formato SPI), con el propósito de almacenar datos en caso que fuera necesario. Las cuatro líneas del protocolo son implementadas en los puertos GPIO 16, 17, 18 y 19 respectivamente.\\

\subsubsection{Puerto JTAG}

JTAG, del inglés \textit{Joint Test Action Group}, es un estándar industrial que implementa un método integrado para verificar diseños y comprobar conexiones de circuitos impresos luego de la fabricación. También incluye funcionalidad de programación en el sistema o \textit{in-system programming} que permite programar un dispositivo ya instalado en un sistema completo, y capacidad de realizar debugging de firmware.\\

La representación física de este protocolo es un puerto, a través del cual el dispositivo se conecta a un programador y a una computadora que permite cargar programas y realizar debugging. Para los controladores de la serie C2000 se implementa un puerto JTAG de 7x2 pines, los cuáles están asignados a los últimos terminales del conector DIMM-100 de la controlCARD.\\

\subsubsection{Comunicación por USB}

Para implementar la comunicación por puerto tipo USB-B, como ya se mencionó, se utiliza el protocolo serie UART. Sin embargo, los formatos de datos de USB y UART son distintos, por lo que de debe implementar algún tipo de circuito que convierta y adapte el formato de datos entre ambos protocolos. Para cumplir esta función, se utilizó un circuito integrado USB-UART modelo {\Medium FT232BL de FTDI Chip} como el de la figura \ref{ft232bl}, que implementa una variedad de funciones y tiene salida USB compatible con todos los sistemas operativos importantes.\\

\begin{figure}[h]
    \centering
    \includegraphics[scale=0.2]{Imagenes/FT232BL.jpg}
    \caption{Convertidor USB-UART para la implementación de una conexión USB, modelo FT232BL de FTDI Chip, en su encapsulado LQFP-32.}
    \label{ft232bl}
\end{figure}

Según indicación de la hoja de datos del fabricante, se colocó un oscilador de cristal de \SI[]{6}{\mega\hertz} de dos pines que requiere dos capacitores de carga adicionales. Adicionalmente, se implementaron LEDs para indicar la recepción y transmisión de datos, y un circuito para el reset del integrado utilizando un pulsador.\textsuperscript{\cite{DatasheetFT232BL}}\\

Además, se implementaron circuitos de alimentación de USB, independientes de la alimentación externa de \SI[]{12}{\volt} de la plataforma, con una línea de \SI[]{5}{\volt} proveniente directamente del puerto USB y una línea de \SI[]{3.3}{\volt} derivada de la alimentación USB mediante un regulador lineal LP2985-3.3 como el que se utilizó para la alimentación de \SI[]{3.3}{\volt} de la parte digital del sistema.\\

Finalmente, se agregó un circuito de \textit{soft start} o arranque suave, utilizando un MOSFET de canal P modelo IRF7425, un capacitor de \SI[]{0.1}{\micro\farad} y un resistor de \SI[]{1}{\kilo\ohm}, para limitar los picos de corriente en encendido, que podrían ser lo suficientemente grandes como para resetar el FT232BL.\\

\subsubsection{Entrada/Salida General}

Además de todos estos periféricos, se utilizaron algunos puertos GPIO libres para implementar algún tipo de interfaz con el usuario: los puertos GPIO 49, 59, 61 y 63 se conectan cada uno a un LED de montaje superficial y se configuran como salida; y los puertos GPIO 5, 7 , 9 y 11 se conectan a cuatro pulsadores y se configuran como puerto de entrada.\\

También se colocó una tira de pines extra que conecta a los puertos GPIO 13, 14, 58, 60, 62 y 85 sin utilizar, para dejarlos disponibles en caso de que se quisiera implementar alguna funcionalidad de entrada/salida adicional por fuera de la plataforma.\\

\newpage

\subsection{Circuito de Alimentación}