\subsection{Sistema de Control}

Ahora, se deben diseñar los circuitos auxiliares del sistema de control que rodean al controlador digital de señales. Estos son mayormente circuitos muy sencillos, ya que la gran parte de los circuitos que implementan las funcionalidades de periféricos se encuentran incluidos en el paquete de evaluación (controlCARD, de la figura \ref{ControlCARD}) del DSC.\\

Para el diseño de estos circuitos de la plataforma correspondiente al sistema de control, se utilizó como referencia una placa adaptadora para la controlCARD del controlador TMS320F28335 que se encuentra en el laboratorio. Esta placa es utilizada para otros proyecto de control del laboratorio, y hace uso de otras funcionalidades adicionales del DSC, como el módulo dedicado para el uso de encoders. Se decidió por basar el sistema de control en este diseño, principalmente porque ya ha tenido extenso uso para control y es comprobado que funciona correctamente.\\

\subsubsection{Implementación de Periféricos}

El módulo de controlCARD en el que se encuentra el TMS320F28335 se conecta mediante la interfaz DIMM-100 de 100 pines. Algunos de estos pines son dedicados a la alimentación y conexión a tierra del dispositivo, interfaz JTAG (que trataremos ahora) y puertos de conversor analógico-digital, con el restante de los pines mapeados a distintos puertos de entrada-salida de propósito general (GPIOxx).\\

\begin{figure}[h]
    \centering
    \includegraphics[scale=0.5]{Imagenes/DIMM100.png}
    \caption{Socket de conexión de cien pines de tipo DIMM-100, donde se inserta la controlCARD.}
    \label{dimm100}
\end{figure}

Estos pines de GPIO pueden ser configurados mediante un registro del DSC para cambiar su funcionalidad: cada pin cuenta con un multiplexor, de manera que, por ejemplo, el pin GPIO 0 puede funcionar como pin de entrada/salida general, o bien como salida de la señal ePWM1A del módulo ePWM. Entonces, si observamos el \textit{pinout} de la controlCARD, podemos decidir cuáles pines del conector asignar para cada funcionalidad.\\

\paragraph{Salidas ePWM}

Como se mencionó en el capítulo \ref{analisis}, el TMS320F28335 cuenta con seis módulos ePWM, cada uno con dos salidas, A y B, es decir doce salidas totales. La controlCARD tiene pines asignados para cada una de estas salidas. En nuestro caso vamos a usar únicamente los primeros cuatro pines, correspondientes a las salidas ePWM1A, ePWM1B, ePWM2A y ePWM2B (Pines GPIO 0, 1, 2 y 3).\\

Estos pines se van a utilizar de manera que las dos salidas del módulo ePWM1 controlan la pata izquierda del puente, y las dos del módulo ePWM2 controlar la pata derecha. Las dos salidas de cada módulo se configuran para estar en contrafase, con un dead-time agregado entre estas señales de \SI[]{200}{\nano\second}. Luego, internamente se conecta la salida de sincronización SYNCO del primer módulo a la entrada de sincronización SYNCI del segundo módulo, y mediante el registro de fase del segundo módulo se controla su fase relativa a ePWM1, logrando una implementación del control de ciclo de trabajo mediante phase-shift que se explicó en la descripción del convertidor.\\

Las señales ePWM utilizadas, junto con las señales ePWM3A y ePWM4A (GPIO 4 y 6) del tercer módulo del DSC son conectadas también a una tira de pines para facilitar la medición de estas señales con un osciloscopio. Además, las señales del tercer módulo se pueden utilizar para implementar una funcionalidad adicional por fuera de los componentes de la plataforma.\\

\paragraph{Entradas de ADC}

Para la entrada de datos proveniente del sensor de efecto hall TMCS1100A4, se utiliza el módulo del conversor analógico digital de 12 bits y \SI[]{80}{\nano\second} de tiempo de conversión que se describió en el capítulo \ref{analisis}. Nuevamente, la controlCARD tiene pines asignados para cada una de las 16 (8 por cada canal A y B) entradas del conversor analógico-digital.\\

Para los datos del sensor se utiliza la primer entrada del canal A, ADCIN-A0, y se conectó una tira de pines a las tres primeras entradas del convertidor, hasta ADCIN-A2. Esto cumple el mismo propósito que la tira de pines para las señales ePWM.\\

\paragraph{Comunicación Serie}

Todos los controladores de la serie C2000 implementan cuatro protocolos de comunicación serie distintos: un bus {\Medium I\textsuperscript{2}C} (\textit{Inter-Integrated Circuit}), una interfaz {\Medium SPI} (\textit{Serial Peripheral Interface}), un módulo {\Medium SCI} (\textit{Serial Communications Interface}) más conocido como UART, y un módulo {\Medium McBSP} (\textit{Multichannel Buffered Serial Port}). El único protocolo que no se va a utilizar en esta plataforma es el de McBSP.\\

\subparagraph{I\textsuperscript{2}C}

El módulo I\textsuperscript{2}C, utilizado para transmitir los datos de tensión, corriente y potencia que entrega el LM5056A, se describió en detalle en el capítulo del sistema de medición. Las líneas de datos SDA y de reloj SCL estan multiplexadas a los pines GPIO 32 y 33 del controlador respectivamente. El cálculo de las resistencias de pull-up y el circuito de separación de la línea de datos ya se trató en el análisis detallado del protocolo.\\

Ambas líneas del bus también tienen salida a través de una tira de pines, para facilitar la evaluación del funcionamiento del bus, y conectar dispositivos adicionales, si se llegara a requerir.\\

\subparagraph{SCI (UART)}

El protocolo SCI, mejor conocido como UART (\textit{Universal Asynchronous Reciever-Transmitter}), es un bus de transmisión de datos en serie, asincrónico (sin señal de reloj para sincronizar), y full-duplex (capacidad de transmitir datos en ambas direcciones simultáneamente). Es un protocolo de velocidad de transmisión variable, y es común su utilización para la comunicación entre un controlador y dispositivos periféricos.\\

El TMS320F28335 cuenta con tres módulos independientes para comunicación por SCI (módulos A, B y C). En la plataforma se utilizó un único módulo para la transmisión de datos mediante el puerto USB. Esto se implementó mediante un integrado que convierte los datos que ingresan por el puerto a un formato para ser transmitido por UART. Las líneas de recepción SCIRX-A y transmisión SCITX-A del primer módulo corresponden a los puertos GPIO 28 y 29 respectivamente. Adicionalmente, al igual que el I\textsuperscript{2}C, se conectaron ambas líneas a una tira de pines.\\

\subparagraph{SPI}

La interfaz SPI es un protocolo de transmisión de datos en formato serie, sincrónico y full-duplex desarrollado por Motorola, y es generalmente utilizado para comunicaciones de corta distancia en sistemas embebidos. Es un bus multi-dispositivo y de esquema maestro/esclavo que cuenta con cuatro lineas: SPI-SIMO, para transmisión de datos del maestro hacia el esclavo; SPI-MISO, para transmisión de datos del esclavo hacia el maestro; SPI-CLK para transmitir la señal de reloj; y SPI-STE, que implementa un \textit{chip-select} para seleccionar el esclavo correcto.\\

En el controlador, este protocolo se implementa para conectar un módulo para lectura y escritura de tarjeta SD (se suelen vender módulos completos que incluyen la traducción de datos al formato SPI), con el propósito de almacenar datos en caso que fuera necesario. Las cuatro líneas del protocolo son implementadas en los puertos GPIO 16, 17, 18 y 19 respectivamente.\\

\subsubsection{Puerto JTAG}

JTAG, del inglés \textit{Joint Test Action Group}, es un estándar industrial que implementa un método integrado para verificar diseños y comprobar conexiones de circuitos impresos luego de la fabricación. También incluye funcionalidad de programación en el sistema o \textit{in-system programming} que permite programar un dispositivo ya instalado en un sistema completo, y capacidad de realizar debugging de firmware.\\

La representación física de este protocolo es un puerto, a través del cual el dispositivo se conecta a un programador y a una computadora que permite cargar programas y realizar debugging. Para los controladores de la serie C2000 se implementa un puerto JTAG de 7x2 pines, los cuáles están asignados a los últimos terminales del conector DIMM-100 de la controlCARD.\\

\subsubsection{Comunicación por USB}

Para implementar la comunicación por puerto tipo USB-B, como ya se mencionó, se utiliza el protocolo serie UART. Sin embargo, los formatos de datos de USB y UART son distintos, por lo que de debe implementar algún tipo de circuito que convierta y adapte el formato de datos entre ambos protocolos. Para cumplir esta función, se utilizó un circuito integrado USB-UART modelo {\Medium FT232BL de FTDI Chip} como el de la figura \ref{ft232bl}, que implementa una variedad de funciones y tiene salida USB compatible con todos los sistemas operativos importantes.\\

\begin{figure}[h]
    \centering
    \includegraphics[scale=0.2]{Imagenes/FT232BL.jpg}
    \caption{Convertidor USB-UART para la implementación de una conexión USB, modelo FT232BL de FTDI Chip, en su encapsulado LQFP-32.}
    \label{ft232bl}
\end{figure}

Según indicación de la hoja de datos del fabricante, se colocó un oscilador de cristal de \SI[]{6}{\mega\hertz} de dos pines que requiere dos capacitores de carga adicionales. Adicionalmente, se implementaron LEDs para indicar la recepción y transmisión de datos, y un circuito para el reset del integrado utilizando un pulsador.\textsuperscript{\cite{DatasheetFT232BL}}\\

Además, se implementaron circuitos de alimentación de USB, independientes de la alimentación externa de \SI[]{12}{\volt} de la plataforma, con una línea de \SI[]{5}{\volt} proveniente directamente del puerto USB y una línea de \SI[]{3.3}{\volt} derivada de la alimentación USB mediante un regulador lineal LP2985-3.3 como el que se utilizó para la alimentación de \SI[]{3.3}{\volt} de la parte digital del sistema.\\

Finalmente, se agregó un circuito de \textit{soft start} o arranque suave, utilizando un MOSFET de canal P modelo IRF7425, un capacitor de \SI[]{0.1}{\micro\farad} y un resistor de \SI[]{1}{\kilo\ohm}, para limitar los picos de corriente en encendido, que podrían ser lo suficientemente grandes como para resetar el FT232BL.\\

\subsubsection{Entrada/Salida General}

Además de todos estos periféricos, se utilizaron algunos puertos GPIO libres para implementar algún tipo de interfaz con el usuario: los puertos GPIO 49, 59, 61 y 63 se conectan cada uno a un LED de montaje superficial y se configuran como salida; y los puertos GPIO 5, 7 , 9 y 11 se conectan a cuatro pulsadores y se configuran como puerto de entrada.\\

También se colocó una tira de pines extra que conecta a los puertos GPIO 13, 14, 58, 60, 62 y 85 sin utilizar, para dejarlos disponibles en caso de que se quisiera implementar alguna funcionalidad de entrada/salida adicional por fuera de la plataforma.\\